\documentclass[a4paper,
fontsize=11pt,
%headings=small,
oneside,
numbers=noperiodatend,
parskip=half-,
bibliography=totoc,
final
]{scrartcl}

\usepackage[babel]{csquotes}
\usepackage{synttree}
\usepackage{graphicx}
\setkeys{Gin}{width=.4\textwidth} %default pics size

\graphicspath{{./plots/}}
\usepackage[ngerman]{babel}
\usepackage[T1]{fontenc}
%\usepackage{amsmath}
\usepackage[utf8x]{inputenc}
\usepackage [hyphens]{url}
\usepackage{booktabs} 
\usepackage[left=2.4cm,right=2.4cm,top=2.3cm,bottom=2cm,includeheadfoot]{geometry}
\usepackage[labelformat=empty]{caption} % option 'labelformat=empty]' to surpress adding "Abbildung 1:" or "Figure 1" before each caption / use parameter '\captionsetup{labelformat=empty}' instead to change this for just one caption
\usepackage{eurosym}
\usepackage{multirow}
\usepackage[ngerman]{varioref}
\setcapindent{1em}
\renewcommand{\labelitemi}{--}
\usepackage{paralist}
\usepackage{pdfpages}
\usepackage{lscape}
\usepackage{float}
\usepackage{acronym}
\usepackage{eurosym}
\usepackage{longtable,lscape}
\usepackage{mathpazo}
\usepackage[normalem]{ulem} %emphasize weiterhin kursiv
\usepackage[flushmargin,ragged]{footmisc} % left align footnote
\usepackage{ccicons} 
\setcapindent{0pt} % no indentation in captions

%%%% fancy LIBREAS URL color 
\usepackage{xcolor}
\definecolor{libreas}{RGB}{112,0,0}

\usepackage{listings}

\urlstyle{same}  % don't use monospace font for urls

\usepackage[fleqn]{amsmath}

%adjust fontsize for part

\usepackage{sectsty}
\partfont{\large}

%Das BibTeX-Zeichen mit \BibTeX setzen:
\def\symbol#1{\char #1\relax}
\def\bsl{{\tt\symbol{'134}}}
\def\BibTeX{{\rm B\kern-.05em{\sc i\kern-.025em b}\kern-.08em
    T\kern-.1667em\lower.7ex\hbox{E}\kern-.125emX}}

\usepackage{fancyhdr}
\fancyhf{}
\pagestyle{fancyplain}
\fancyhead[R]{\thepage}

% make sure bookmarks are created eventough sections are not numbered!
% uncommend if sections are numbered (bookmarks created by default)
\makeatletter
\renewcommand\@seccntformat[1]{}
\makeatother

% typo setup
\clubpenalty = 10000
\widowpenalty = 10000
\displaywidowpenalty = 10000

\usepackage{hyperxmp}
\usepackage[colorlinks, linkcolor=black,citecolor=black, urlcolor=libreas,
breaklinks= true,bookmarks=true,bookmarksopen=true]{hyperref}
\usepackage{breakurl}

%meta
%meta

\fancyhead[L]{M. Limpinsel-Pesavento\\ %author
LIBREAS. Library Ideas, 43 (2023). % journal, issue, volume.
\href{https://doi.org/10.18452/...}{\color{black}https://doi.org/10.18452/...}
{}} % doi 
\fancyhead[R]{\thepage} %page number
\fancyfoot[L] {\ccLogo \ccAttribution\ \href{https://creativecommons.org/licenses/by/4.0/}{\color{black}Creative Commons BY 4.0}}  %licence
\fancyfoot[R] {ISSN: 1860-7950}

\title{\LARGE{Welche Funktion erfüllt die Bibliothek der Zukunft? Bibliotheken als wirtschaftliche Systeme}}% title
\author{Mirco Limpinsel-Pesavento} % author

\setcounter{page}{1}

\hypersetup{%
      pdftitle={Welche Funktion erfüllt die Bibliothek der Zukunft? Bibliotheken als wirtschaftliche Systeme},
      pdfauthor={Mirco Limpinsel-Pesavento},
      pdfcopyright={CC BY 4.0 International},
      pdfsubject={LIBREAS. Library Ideas, 43 (2023).},
      pdfkeywords={Bibliotheken; Systemtheorie; Wirtschaftssystem; Informationskompetenz},
      pdflicenseurl={https://creativecommons.org/licenses/by/4.0/},
      pdfurl={https://doi.org/10.18452/...},
      pdfdoi={10.18452/...},
      pdflang={de},
      pdfmetalang={de}
     }



\date{}
\begin{document}

\maketitle
\thispagestyle{fancyplain} 

%abstracts
\begin{abstract}
\noindent
\textbf{Kurzfassung}: Der Beitrag geht hervor aus einer
bibliothekswissenschaftlichen Abschlussarbeit, in der es um
bibliothekarische Organisationsentwicklung aus system\-theoretischer
Perspektive ging. Anstatt die Arbeit langweilig zusammenzufassen, wird
ein Aspekt diskutiert, der sich aus der systemtheoretischen Annahme
ergibt, dass es sich bei Bibliotheken um Teile des Wirtschaftssystems
handelt. Insbesondere vor dem Hintergrund der Digitalisierungskrise
lassen sich aus dieser Perspektive instruktive Impulse ableiten.
\end{abstract}

%body
Der vorliegende Beitrag geht auf eine im Jahr 2021 entstandene
Abschlussarbeit am Berliner \emph{Institut für Bibliotheks- und
Informationswissenschaft} zurück. In dieser unternahm der Verfasser den
Versuch, die bibliothekarische Organisationsentwicklung vor dem
Hintergrund der soziologischen Systemtheorie zu modellieren.\footnote{Mirco
  Limpinsel-Pesavento: \emph{Systemtheorie als Heuristik für die
  bibliothekarische Strategie- und Organisationsentwicklung}, Berlin
  2022 (= Berliner Handreichungen zur Bibliotheks- und
  Informationswissenschaft, Heft 489), DOI:
  \url{https://doi.org/10.18452/24411}.} In der Abschlussarbeit wurden
auch systemtheoretische Grundbegriffe eingeführt und die Frage erörtert,
ob und inwiefern der Gegenstand Bibliothek überhaupt als System
konzipiert werden kann -- eine Frage, die kaum befriedigend beantwortet
werden konnte. Erst im Laufe der Untersuchungen entstand die Idee für
eine womöglich kontraintuitiv wirkende These, der im Folgenden -- ohne
die strenge Form einer Qualifikationsschrift -- etwas ausführlicher
nachgegangen werden soll. Wenn man den Gegenstand Bibliothek aus
systemtheoretischer Perspektive behandelt, liegt die Frage nahe, zu
welchem der großen gesellschaftlichen Funktionssysteme die Bibliothek
gehört. Die Antwort, so die Hypothese, lautet: Bibliotheken gehören zum
Wirtschaftssystem der Gesellschaft.

\hypertarget{bibliotheken-als-teil-des-wirtschaftssystems}{%
\section{Bibliotheken als Teil des
Wirtschaftssystems}\label{bibliotheken-als-teil-des-wirtschaftssystems}}

Ausgerechnet zum Wirtschaftssystem? Die These muss Widerspruch erregen.
Bibliotheken verstehen sich doch gerade als Gegenpol zum
Wirtschaftlichen, stehen ein für Offenheit und Teilhabe. Ein Buch
auszuleihen heißt ja gerade, das Buch \emph{nicht} zu kaufen und Open
Access kann man durchaus als Gegenprogramm zum Informationskapitalismus
verstehen. In den Verhandlungen mit den großen Wissenschaftsverlagen
stehen die Bibliotheken für die freie Verteilung informationeller Güter.
Der Beitrag muss daher zwei Teile bedienen: Erstens muss die These
kontextualisiert und plausibilisiert werden. Zweitens muss sie im
Anschluss diskutiert werden, das heißt, es können einige Konsequenzen
ausbuchstabiert werden, die aus der Betrachtung von Bibliotheken als
Teil des Wirtschaftsgeschehens folgen.

Eine grundlegende soziologische und genauer systemtheoretische
Beschreibung des Bibliothekswesens stand nach einen kleinen Konjunktur
in den 1970er Jahren\footnote{Siehe beispielsweise Gernot Wersig:
  Information -- Kommunikation -- Dokumentation. Ein Beitrag zur
  Orientierung der Informations- und Dokumentationswissenschaften,
  Pullach bei München 1974 sowie Wolfram Henning: \enquote{Öffentliche
  Bibliothek und soziale Kommunikation}, in: Deutscher
  Bibliotheksverband / Arbeitsstelle für das Bibliothekswesen (Hrsg.):
  Bibliothekswissenschaft und öffentliche Bibliothek. Referate und
  Ergebniszusammenfassungen eines Fortbildungsseminars der FHB
  Stuttgart, Berlin 1974 (= Bibliotheksdienst, Beiheft 102/03).} lange
nicht auf der Tagesordnung.\footnote{Vgl. Fabio Tullio: \enquote{Zur
  Legitimation Öffentlicher Bibliotheken}, in: LIBREAS. Library Ideas
  30, 2016. \url{https://libreas.eu/ausgabe30/tullio/}. (13.01.2023).}
Stellt die Bibliothekswissenschaft die praktischen und speziellen
Aspekte bibliothekarischer Arbeit einmal zugunsten eines großen Ganzen
in den Hintergrund, so geschieht dies meist eher unter dem Stichwort
\emph{politischer} Teilhabe.\footnote{Siehe zum Beispiel Hans-Christoph
  Hobohm: \enquote{Bibliotheken und Demokratie in Deutschland.
  Ergebnisse eines europäischen Projektes zur Rolle öffentlicher
  Bibliotheken für Demokratie und Gemeinwohl}, in: o-bib. Das offene
  Bibliotheksjournal 6, Nr. 4 (2019), 8--25,
  \url{https://doi.org/10.5282/o-bib/2019H4S7-24}, sowie Michael M.
  Widdersheim: \enquote{A Political Theory of Public Library
  Development}, in: Libri 68, Nr. 4 (2018), 269--289.
  \url{https://doi.org/10.1515/libri-2018-0024}.} In den vergangenen
Jahren scheint sich indessen das bibliothekswissenschaftliche Interesse
für die Soziologie wieder zu mehren.\footnote{Siehe beispielsweise Ulla
  Wimmer: Die Position der Öffentlichen Bibliotheken im Bibliotheksfeld
  und im bibliothekarischen Fachdiskurs der Bundesrepublik Deutschland
  seit 1964, Diss. Berlin 2019, \url{https://doi.org/10.18452/19791},
  sowie, mit explizitem Bezug auf die Systemtheorie, Hermann
  Rösch/Jürgen Seefeldt/Konrad Umlauf: Bibliotheken und
  Informationsgesellschaft in Deutschland. Eine Einführung.
  Mitbegrüundet von Engelbert Plassmann. 3., neu konzipierte und
  aktualisierte Auflage unter Mitarbeit von Albert Bilo und Eric W.
  Steinhauer, Wiesbaden 2019.} Auch wenn es vielleicht zu früh wäre, von
einer Renaissance der soziologischen Bibliothekswissenschaft zu
sprechen, fällt der abstraktere, nüchterne Bezug auf die
\emph{gesellschaftliche} Dimension der Bibliotheken auf. Es ist
vielleicht kein Zufall, dass diese neue Konjunktur mit einer in letzter
Zeit immer wieder attestierten Krise zusammenfällt.

Die These, dass es instruktiv ist, Bibliotheken als Teil des
Wirtschaftssystems zu betrachten und ihre Funktionsweise und ihr
strategisches Potenzial am Vorbild der Wirtschaft zu verstehen, ist nur
vor dem Hintergrund einer systemtheoretischen Perspektive verständlich.
Damit ist die soziologische Systemtheorie des Bielefelder Soziologen
Niklas Luhmann (1927--1998) gemeint. Der vorliegende Beitrag versucht
aber weder, eine Einführung in die Theorie zu geben, noch setzt er deren
Kenntnis voraus. Für eine Einführung und weiterführende
Literaturhinweise sei auf die Abschlussarbeit verwiesen.\footnote{Siehe
  Fußnote 1. Insbesondere die ersten beiden Kapitel enthalten
  Einführendes zur Systemtheorie und ihrer Anwendbarkeit auf den
  Gegenstand Bibliotheken sowie Hinweise auf Primär- und einführende
  Sekundärliteratur.}


\hypertarget{eine-systemtheoretische-perspektive}{%
\section{Eine systemtheoretische
Perspektive}\label{eine-systemtheoretische-perspektive}}

Niklas Luhmann versteht die moderne Gesellschaft als eine Verflechtung
unterschiedlicher, ja, inkommensurabler Systemperspektiven. Es ist
insofern naheliegend, zu fragen, \enquote{wohin} Bibliothek im Schema
der Systemtheorie \enquote{gehört}: In welches von Luhmann beschriebene
System passt sie? Oder handelt es sich sogar um ein völlig
eigenständiges System, dessen Beschreibung noch aussteht?

Als soziale Organisation ist Bibliothek sicherlich ein \enquote{System}.
Der Begriff System ist jedoch für sich genommen noch zu unspezifisch,
als dass er systemtheoretisch fruchtbar gemacht werden könnte.
Typischerweise fragen Systemtheoretiker:innen nach den so genannten
\emph{Funktionssystemen} der Gesellschaft. Als Funktionssysteme
beschreibt Luhmann die großen Bereiche, deren Logik die gesamte
Gesellschaft durchziehen -- das sind Wirtschaft, Wissenschaft, Recht,
Kunst, Politik, Religion und Erziehung (jedenfalls sind das die
Funktionssysteme, die Luhmann jeweils monografisch beschrieben hat).
Jedes Funktionssystem bearbeitet genau eine Funktion und stellt sie für
die gesamtgesellschaftliche Kommunikation bereit. Beispielsweise erzeugt
das Wissenschaftssystem die Möglichkeit, sich auf Wahrheiten zu
beziehen, das Wirtschaftssystem bearbeitet die Frage, wie knappe Güter
verteilt werden können und das Rechtssystem ermöglicht es, normativ zu
erwarten (also auch dann noch zu erwarten, wenn die Erwartung enttäuscht
wird).

Da fast alle Bücher Luhmanns einzelne Funktionssysteme thematisieren,
scheint es nahe zu liegen, eine \enquote{systemtheoretische}
Beschreibung gleichzusetzen mit einer Beschreibung als Funktionssystem.
Zumindest für Bibliotheken wäre das aber ein Kurzschluss: Zwar sind
Bibliotheken durchaus Systeme. Sie sind jedoch aus einer ganzen Reihe
von Gründen \emph{keine} eigenlogisch operierenden
Funktionssysteme.\footnote{Diese These wird in der Abschlussarbeit
  (siehe Fußnote 1), S. 18 ff., ausführlich diskutiert.} Die Frage
bleibt also: Was \enquote{ist} Bibliothek aus systemtheoretischer Sicht?
Die Antwort ist gar nicht einfach zu formulieren. In der Abschlussarbeit
wurde jedenfalls keine rundum überzeugende Antwort gefunden.

Nimmt man an, dass Bibliotheken dann, wenn sie schon keine eigenen
Funktionssysteme sind, doch vielleicht zu einem der großen
Funktionssysteme \enquote{gehören}, so bleibt unklar, zu welchem. Zur
Wissenschaft? Diese Sichtweise ist in Wissenschaftlichen Bibliotheken
wohl verbreitet, lässt sich aber nicht überzeugend begründen. Zum
Erziehungssystem? Jedenfalls kann man in Bibliotheken viel lernen und
die Wissenschaftlichen Bibliotheken gehören auch größtenteils
organisatorisch zu den Universitäten. Aber was wäre dann mit den
Öffentlichen Bibliotheken? Handelt es sich bei ihnen nicht eher um
Einrichtungen für \emph{politische} Teilhabe? Auch hier gibt es keine
einfache Antwort.\footnote{Diese Diskussion findet sich für eine Reihe
  von Funktionssystemen in der Abschlussarbeit (siehe Fußnote 1), S. 23
  ff.} Im Folgenden soll also die Hypothese angenommen werden, dass
Bibliotheken zum Wirtschaftssystem der Gesellschaft gehören. Sie ist
sicher nicht vollständig zu überprüfen und es mag eine Reihe von Gründen
geben, sie abzulehnen. Dennoch lassen sich aus ihr einige instruktive
Hinweise für die Bibliothekswissenschaft ziehen.

\hypertarget{rolle-im-wirtschaftssystem}{%
\section{Rolle im
Wirtschaftssystem}\label{rolle-im-wirtschaftssystem}}

Was bedeutet es, Bibliotheken als \enquote{Teil} des Wirtschaftssystems
zu verstehen? Zunächst bedeutet es nicht, dass es in der Gesellschaft
eine Art Schublade \enquote{Wirtschaft} gibt, in der die Bibliothek
liegt. Dass Bibliotheken zum Funktionssystem Wirtschaft
\enquote{gehören}, meint vielmehr, dass sie nach einer für dieses System
spezifischen Logik operieren und ein bestimmtes Bezugsproblem
bearbeiten. Dieses Problem ist die \emph{Verteilung knapper
Güter}.\footnote{Eine Diskussion der systemtheoretischen Analyse des
  Konzepts Knappheit soll hier nicht erfolgen. Siehe zu dem Begriff
  Niklas Luhmann: Die Wirtschaft der Gesellschaft, Frankfurt am Main
  1994, 177 ff.} Knapp ist beispielsweise Mehl: Jeder Zugriff auf Mehl
verringert die Menge des noch verfügbaren Mehls, steigert also die
Knappheit; zugleich ist Knappheit ein wesentlicher Grund für den Zugriff
auf das Mehl, denn gerade seine Knappheit erzeugt den Wunsch, sich damit
zu bevorraten, was die Knappheit weiter steigert. Diese Paradoxie und
die kommunikativen Mechanismen ihrer Bearbeitung sind das Bezugsproblem,
das laut Luhmann zur Ausdifferenzierung eines eigenlogisch operierenden
Wirtschaftssystems geführt hat. Auch Informationen waren in der
Wirtschaft, wie wir sie bis vor wenigen Jahren kannten, prinzipiell
knappe Güter. Ihre Träger waren Bücher, die aufwändig produziert,
gedruckt und distribuiert werden mussten, die vergriffen sein konnten
und die entsprechend ihren Preis hatten.

Ein weiteres essentielles Merkmal des Wirtschaftssystems nach Luhmann
sind \emph{Zahlungen}; sie leiten sämtliche Operationen im Sinne einer
Leitunterscheidung an. Wäre es aber nicht zu kurz gegriffen,
Bibliotheken auf Knappheit und Zahlungen zu reduzieren? Der Leihverkehr
ist doch eigentlich ein Gegenentwurf zur Wirtschaft und kommt, sieht man
einmal von etwaigen Mahngebühren ab, komplett zahlungsfrei aus?

Es ist gerade spezifisch für einen \emph{soziologischen} Blick auf die
Dinge, eine Distanz einzunehmen, aus der heraus auf den ersten Blick
sehr ungleiche Dinge im Hinblick auf ihre Funktion vergleichbar werden.
Denn natürlich leisten Bibliotheken permanent Zahlungen. Als eine Art
Zwischenglied zwischen Verlagen und Benutzer:innen kaufen sie Verlagen
Bücher ab und stellen sie für die Benutzung bereit. So betrachtet,
scheint die Hypothese, dass es sich bei Bibliotheken im Wesentlichen um
\emph{wirtschaftliche} Einrichtungen handelt, gar nicht mehr so abwegig.
Die weiteren Angebote und Leistungen der Bibliotheken kann man als
Infrastrukturen beschreiben, die diese Kernoperation vermitteln: Die
gekauften Ressourcen werden dann nicht nur für die Benutzung beschafft,
sondern beispielsweise auch erschlossen, es werden Räume angeboten, in
denen die Nutzung stattfinden kann, und es finden allerlei logistische
Zusatzanstrengungen statt, die den ganzen Ablauf überhaupt ermöglichen.
Zugleich haben die Bibliotheken um die Medienbenutzung herum ein
umfassendes Beratungsangebot aufgebaut. Dies ist allerdings nicht
spezifisch für Bibliotheken. Auch ein Supermarkt \enquote{macht}
beispielsweise sehr viel mehr, als nur Geld an der Kasse einzusammeln.
Das Sortiment muss kuratiert, die Waren bestellt, Lieferabläufe
koordiniert, Regale eingeräumt werden, um nur einige Parallelen zu
nennen.

Man kann festhalten: Bibliotheken organisieren die, nach Luhmann,
Leitunterscheidung des Wirtschaftssystems von \emph{haben} und
\emph{nicht haben} für einen spezifischen Bereich -- Informationen --
und auf spezifische Weise, nämlich so, dass die Benutzer:innen, die
deshalb auch meist nicht Kund:innen heißen, dafür nicht direkt zahlen.

Warum soll so eine Umschreibung (außer, dass sie soziologisch ist)
interessant sein? Nicht so sehr deshalb, weil sie Einblick in die wahre
Natur der Bibliotheken böte. Dafür wäre die Argumentation auch viel zu
kurz und zu wenig historisch. Interessant ist diese Perspektive vielmehr
deshalb, weil sie ein neues Licht auf einige Konsequenzen wirft. Die
Frage, welche Funktion die Bibliotheken in Zukunft (noch) erfüllen
können oder sollen, bekommt durch sie eine neue Form und einen neuen
Beantwortungshorizont.

\hypertarget{transformation-und-funktionskrise}{%
\section{Transformation und
Funktionskrise}\label{transformation-und-funktionskrise}}

So kursorisch nun die systemtheoretische Hypothese entwickelt wurde, so
knapp sollen im Folgenden einige dieser Konsequenzen besprochen werden,
und zwar zunächst nur für die Wissenschaftlichen Bibliotheken.
Ausgangspunkt ist die Beobachtung, dass es im Zuge der von manchen so
genannten digitalen Transformation der Gesellschaft seit einiger Zeit zu
massiven Transformationen des Bezugsproblems der Knappheit an
informationellen Gütern kommt.\footnote{Zur universellen Tragweite der
  digitalen Transformationen siehe nur Michel Serres: Erfindet euch neu!
  Eine Liebeserklärung an die vernetzte Generation, dt. Übers. Frankfurt
  am Main 2013.} Diese Transformationen führen die Bibliotheken in eine
\emph{Funktionskrise}, also in eine Unklarheit darüber, ob die bisherige
Funktion der Bibliotheken für zukünftige Gesellschaft noch angemessen
sein kann. Die Auswirkungen dieser Krise kündigen sich bereits seit
einigen Jahrzehnten beispielsweise in den ungebrochenen Beschwörungen
einer goldenen Bibliothekszukunft\footnote{Siehe dazu Stefan Gradmann:
  \enquote{Die Bibliothek der Zukunft}, in: Konrad Umlauf/Stefan
  Gradmann (Hrsg.): Handbuch Bibliothek. Geschichte, Aufgaben,
  Perspektiven, Heidelberg 2012, 387--397,
  \url{https://doi.org/10.1007/978-3-476-05185-1_10} sowie Michael
  Knoche: Die Idee der Bibliothek und ihre Zukunft, Göttingen 2018.}
oder im grassierenden Neuerfindungs- und Reformgeist\footnote{Siehe
  beispielsweise R. David Lankes: The Atlas of New Librarianship,
  Cambridge, Mass. 2011.} an. Die Situation, die nun Funktionskrise
genannt wurde, bildet das Bezugsproblem für die nun folgende Diskussion
einiger Konsequenzen aus dem Angebot, Bibliotheken als Einrichtungen zu
verstehen, die einer letztlich wirtschaftlichen Funktion dienen.

Was also ändert sich in der Umwelt der Bibliotheken? Zum einen scheint
mit den so genannten Schattenbibliotheken eine Situation entstanden zu
sein, in der ein Zugriff auf die Güter keine Zunahme von Knappheit mehr
zur Folge hat. Die Auswirkungen spüren vor allem die Verlage. Zugleich
eröffnen neue Kommunikationskanäle im Internet und in den sozialen
Netzwerken Wissenschaftler:innen die Möglichkeit, gleichsam an den
Verlagen vorbei, also knappheitsfrei zu publizieren. Für die Suche nach
Informationen entstehen ebenfalls funktionale Äquivalente: Wollte man
früher wissen, wer beispielsweise die Hauptrolle in einem bestimmten
Film gespielt hat, musste man dafür in der Regel eine Bibliothek
aufsuchen. Mit der Möglichkeit, beliebige Informationen schneller und
vollständiger zu finden, als es zuvor je möglich war, bleibt den
Bibliotheken nur noch der argumentative Rückzugsort einer besseren
Informationsqualität -- was sicher heute auch noch weitgehend stimmt.
Dennoch mag man sich fragen, ob sich damit das Funktionsmonopol, das
Bibliotheken über Jahrtausende hatten, auch in Zukunft wird halten
lassen. Die systemtheoretische Perspektive verdeutlicht: In dem Maße, in
dem die digitalen Transformationen die Knappheit informationeller Güter
selbst betreffen, betreffen sie unmittelbar auch die Funktionskrise der
Bibliotheken.

Doch die systemtheoretische Perspektive macht die Funktionskrise nicht
nur verständlich. Sie eignet sich auch für die Suche nach
Lösungsansätzen. So liegt eine Umstellung der Strategien der
Bibliotheken nahe. Die großen Wissenschaftlichen Bibliotheken haben sich
längst in eine Richtung bewegt, in der immer weniger Geld für physische
Bücher und immer mehr Geld für Lizenzierungen fließt. Dieser neuere
Ansatz bleibt, zumindest im Sinne der systemtheoretischen Analyse, der
Funktion der Verteilung treu, dies aber unter völlig neuartigen
Umständen der Benutzung. Wie groß die Unterschiede sind, zeigt sich
schon darin, dass es meist kaum mehr möglich ist, die riesigen
Ressourcenpakete inhaltlich zu erschließen. Diese ehemalige
bibliothekarische Kernkompetenz wird inzwischen häufig den Verlagen
überlassen.

\hypertarget{reputation-als-knappes-gut}{%
\section{Reputation als knappes
Gut}\label{reputation-als-knappes-gut}}

Die Wirtschaftsperspektive legt nun eine Frage nahe, die bisher
eigentlich nicht auf dem bibliothekarischen Radar vorkam, nämlich,
welche alternativen Knappheiten als mögliche neue Betätigungsfelder in
Frage kommen. Während die informationellen Güter unter digital
vernetzten Bedingungen weniger knapp werden, wird Aufmerksamkeit dadurch
automatisch zu einem neuen knappen Gut. Das ist ein Gemeinplatz, der
aber für die Bibliotheken interessant werden kann. Zumal in den
Wissenschaften häufig mit Aufmerksamkeit auch Reputation verbunden wird.

Die Bibliotheken könnten sich nun etwa überlegen, ob sie nicht auf dem
dynamischen Markt für Aufmerksamkeit und wissenschaftliche Reputation
viel stärker als früher mitspielen wollen. Das wäre zumindest ein
sinnvolles Komplement zum Engagement für Open Access. Die
DEAL-Verhandlungen beispielsweise belassen die Funktion der Reputation
klassisch den Verlagen und sichern nur den freien Zugang zu den
Ressourcen. Universitätsverlage und Repositorien sind für renommierte
Wissenschaftler:innen aber nur bedingt attraktiv, wenn sie keine Antwort
auf die Frage nach der Reputation finden können. Auch bleibt das
revolutionäre Potential von Ideen wie Open Access doch sehr begrenzt,
wenn eigentlich die Zahlungen nur an eine andere Stelle verschoben
werden.

Die Verlage bekommen ihr Geld wie vorher (bloß nicht mehr von den
Bibliotheken im Tausch gegen Bücher, sondern von den Bibliotheken als
Publikationsgebühr) und alle können die Publikationen lesen. Würden
Bibliotheken dagegen die Verteilung des knappen Guts Reputation als ihre
Funktion sehen, könnte man Formate entwickeln, in denen die Resultate
öffentlich finanzierter Forschung -- und zwar Wissen \emph{und}
Reputation -- tatsächlich auch öffentlich verteilt werden.

Das betrifft nicht nur die Produktion von Publikationen, sondern ebenso
die Infrastruktur der Wissenschaftskommunikation. Wissenschaftler:innen
kommunizieren seit jeher durch ihre Schriften, seit dem 19. Jahrhundert
vornehmlich in wissenschaftlichen Zeitschriften, deren Reputation unter
anderem dadurch garantiert wird, dass die Wissenschaftler:innen sie
selbst herausgeben. In diesem Ökosystem spielen Bibliotheken eine nicht
wegzudenkende Rolle.

Wenn heute ein nicht geringer Teil der Kommunikation über Mastodon (oder
noch über Twitter) stattfindet, so muss diese Veränderung auch für die
Bibliotheken eine Umstellung bedeuten. Das Projekt \emph{perma.cc} des
Harvard Library Innovation Lab ist dafür ein mustergültiges
Beispiel:\footnote{Projektseite: \url{https://social.perma.cc}
  (05.05.23).} Die faktisch auf Twitter stattfindende
Wissenschaftskommunikation wird archivier- und referenzierbar gemacht
und so wieder zum \enquote{normalen} Teil der Wissenschaft. Zeitgemäße
Informationsinfrastrukturen könnten heute über den superschnellen
Lieferverkehr auf dem Campus und Scanaufträge hinausgehen und etwa
digitale Plattformen für die gemeinsame Erzeugung wissenschaftlichen
Wissens umfassen.

Das wäre, systemtheoretisch betrachtet, kein bibliothekarisches Neuland,
sondern einfach eine zeitgemäße Ausprägung des hergebrachten
Funktionsbezugs des Bibliothekswesens. Zeitgemäße Bibliotheken könnten
so ihren \enquote{Markenkern} selbstbewusst und stringent definieren,
anstatt sich als eklektische Portfolios aus allen möglichen Angeboten
und Dienstleistungen zu präsentieren.

\hypertarget{erweiterung-des-begriffs-informationskompetenz}{%
\section{Erweiterung des Begriffs
Informationskompetenz}\label{erweiterung-des-begriffs-informationskompetenz}}

Auch \emph{User Interfaces} und überhaupt die digitale Repräsentation
kognitiver Inhalte erscheinen damit im Zuständigkeitsbereich der
Bibliotheken. Die Einrichtung von Text war lange Zeit ein wirtschaftlich
organisierter Prozess. Wie Text auf der Seite gesetzt wird, ist weit
mehr als ein pragmatisches und auch weit mehr als ein ästhetisches
Detail.

Typografie hatte immer schon eine kognitive, erkenntnisermöglichende
Funktion und es waren die Verlage, die sich die Expert:innen leisteten,
die wussten, wie man die Bücher als \enquote{perfekte Lesemaschinen}
einrichtet.\footnote{Siehe dazu Roland Reuß: Die perfekte Lesemaschine.
  Zur Ergonomie des Buches, Göttingen: Wallstein 2014.} In dem Maße
aber, in dem Texte direkt auf dem Computer entstehen, machen immer mehr
Verlage die Autor:innen selbst zu Schriftsetzer:innen. Das Resultat:
Bücher, die aussehen wie ein Word-Dokument. Oder die Texte werden gleich
ganz ohne Verlag veröffentlicht; oft fehlt dabei das Bewusstsein, was es
eigentlich heißt, einen Text einzurichten, weil es nämlich nirgends
unterrichtet wird. Wenn Bibliotheken die Knappheit, die sie
organisieren, nicht nur in den Texten selbst sehen würden, sondern etwa
auch in der Kompetenz ihrer ergonomischen Einrichtung (also in ihrer
Benutzbarkeit), würden sie eine ganz neue Expertise hinzugewinnen. Das
wäre gleichsam die andere Seite der Informationskompetenz, Kompetenz
nämlich nicht nur für die Rezeption, sondern auch für die Verbreitung
von Informationen. Dies wäre dann aber keine Neuerfindung und auch kein
Nebenschauplatz, sondern dieselbe Funktion, der sie sich immer schon
widmen: Die Verteilung von knappen Gütern im Bereich der
(wissenschaftlichen) Information.

\hypertarget{knappheit-und-bibliotheken}{%
\section{\texorpdfstring{\enquote{Knappheit} und
Bibliotheken}{``Knappheit'' und Bibliotheken}}\label{knappheit-und-bibliotheken}}

Dies sind nur einige der Möglichkeiten, die sich zeigen, wenn man das
Bezugsproblem bibliothekarischer Arbeit systemtheoretisch als Knappheit
versteht. Als Beitrag zum Themenheft \enquote{Soziologie der Bibliothek}
wurde damit ein Aspekt angerissen, der dazu anregen sollte, neue
Perspektiven auf einen alten Gegenstand auszuprobieren. Wer es lieber
soziologisch-systemtheoretisch präziser und belegter mag, der sei erneut
auf die online zugängliche Abschlussarbeit verwiesen. Dieser Beitrag ist
weniger eine Zusammenfassung der Arbeit als der Versuch, einen
speziellen Aspekt weiterzudenken und damit einen Beitrag zum Nachdenken
über die Zukünfte bibliothekarischer Funktion zu leisten.

%autor
\begin{center}\rule{0.5\linewidth}{0.5pt}\end{center}

\textbf{Mirco Limpinsel-Pesavento} ist Literaturwissenschaftler,
arbeitete unter anderem zur Hermeneutikgeschichte, zur Methodologie der
Digital Humanities sowie zur Architekturgeschichte und ist seit 2019
Bibliothekar, derzeit am Bauhaus-Archiv / Museum für Gestaltung in
Berlin. ORCID: \url{https://orcid.org/0000-0002-4301-6892}

\end{document}