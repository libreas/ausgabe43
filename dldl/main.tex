\documentclass[a4paper,
fontsize=11pt,
%headings=small,
oneside,
numbers=noperiodatend,
parskip=half-,
bibliography=totoc,
final
]{scrartcl}

\usepackage[babel]{csquotes}
\usepackage{synttree}
\usepackage{graphicx}
\setkeys{Gin}{width=.4\textwidth} %default pics size

\graphicspath{{./plots/}}
\usepackage[ngerman]{babel}
\usepackage[T1]{fontenc}
%\usepackage{amsmath}
\usepackage[utf8x]{inputenc}
\usepackage [hyphens]{url}
\usepackage{booktabs} 
\usepackage[left=2.4cm,right=2.4cm,top=2.3cm,bottom=2cm,includeheadfoot]{geometry}
\usepackage[labelformat=empty]{caption} % option 'labelformat=empty]' to surpress adding "Abbildung 1:" or "Figure 1" before each caption / use parameter '\captionsetup{labelformat=empty}' instead to change this for just one caption
\usepackage{eurosym}
\usepackage{multirow}
\usepackage[ngerman]{varioref}
\setcapindent{1em}
\renewcommand{\labelitemi}{--}
\usepackage{paralist}
\usepackage{pdfpages}
\usepackage{lscape}
\usepackage{float}
\usepackage{acronym}
\usepackage{eurosym}
\usepackage{longtable,lscape}
\usepackage{mathpazo}
\usepackage[normalem]{ulem} %emphasize weiterhin kursiv
\usepackage[flushmargin,ragged]{footmisc} % left align footnote
\usepackage{ccicons} 
\setcapindent{0pt} % no indentation in captions

%%%% fancy LIBREAS URL color 
\usepackage{xcolor}
\definecolor{libreas}{RGB}{112,0,0}

\usepackage{listings}

\urlstyle{same}  % don't use monospace font for urls

\usepackage[fleqn]{amsmath}

%adjust fontsize for part

\usepackage{sectsty}
\partfont{\large}

%Das BibTeX-Zeichen mit \BibTeX setzen:
\def\symbol#1{\char #1\relax}
\def\bsl{{\tt\symbol{'134}}}
\def\BibTeX{{\rm B\kern-.05em{\sc i\kern-.025em b}\kern-.08em
    T\kern-.1667em\lower.7ex\hbox{E}\kern-.125emX}}

\usepackage{fancyhdr}
\fancyhf{}
\pagestyle{fancyplain}
\fancyhead[R]{\thepage}

% make sure bookmarks are created eventough sections are not numbered!
% uncommend if sections are numbered (bookmarks created by default)
\makeatletter
\renewcommand\@seccntformat[1]{}
\makeatother

% typo setup
\clubpenalty = 10000
\widowpenalty = 10000
\displaywidowpenalty = 10000

\usepackage{hyperxmp}
\usepackage[colorlinks, linkcolor=black,citecolor=black, urlcolor=libreas,
breaklinks= true,bookmarks=true,bookmarksopen=true]{hyperref}
\usepackage{breakurl}

%meta
\expandafter\def\expandafter\UrlBreaks\expandafter{\UrlBreaks%  save the current one
  \do\a\do\b\do\c\do\d\do\e\do\f\do\g\do\h\do\i\do\j%
  \do\k\do\l\do\m\do\n\do\o\do\p\do\q\do\r\do\s\do\t%
  \do\u\do\v\do\w\do\x\do\y\do\z\do\A\do\B\do\C\do\D%
  \do\E\do\F\do\G\do\H\do\I\do\J\do\K\do\L\do\M\do\N%
  \do\O\do\P\do\Q\do\R\do\S\do\T\do\U\do\V\do\W\do\X%
  \do\Y\do\Z}
%meta

\fancyhead[L]{Redaktion LIBREAS\\ %author
LIBREAS. Library Ideas, 43 (2023). % journal, issue, volume.
\href{https://doi.org/10.18452/27073}{\color{black}https://doi.org/10.18452/27073}
{}} % doi 
\fancyhead[R]{\thepage} %page number
\fancyfoot[L] {\ccLogo \ccAttribution\ \href{https://creativecommons.org/licenses/by/4.0/}{\color{black}Creative Commons BY 4.0}}  %licence
\fancyfoot[R] {ISSN: 1860-7950}

\title{\LARGE{Das liest die LIBREAS, Nummer \#12 (Frühling bis Sommer 2023)}}% title
\author{Redaktion LIBREAS} % author

\setcounter{page}{1}

\hypersetup{%
      pdftitle={Das liest die LIBREAS, Nummer \#12 (Frühling bis Sommer 2023)},
      pdfauthor={Redaktion LIBREAS},
      pdfcopyright={CC BY 4.0 International},
      pdfsubject={LIBREAS. Library Ideas, 43 (2023)},
      pdfkeywords={Literaturübersicht, Bibliothekswissenschaft, Informationswissenschaft, Bibliothekswesen, Rezension, literature overview, library science, information science, library sector, review},
      pdflicenseurl={https://creativecommons.org/licenses/by/4.0/},
      pdfcontacturl={http://libreas.eu},
      baseurl={},
      pdflang={de},
      pdfmetalang={de},
      pdfdoi={10.18452/27073},
      pdfurl={https://doi.org/10.18452/27073}
     }



\date{}
\begin{document}

\maketitle
\thispagestyle{fancyplain} 

%abstracts

%body
Beiträge von Ben Kaden (bk), Sara Juen (sj), Karsten Schuldt (ks), Eva
Bunge (eb), Viola Voß (vv), Dorothea Strecker (ds)

\hypertarget{zur-kolumne}{%
\section{1. Zur Kolumne}\label{zur-kolumne}}

Ziel dieser Kolumne ist es, eine Übersicht über die in der letzten Zeit
erschienene bibliothekarische, informations- und
bibliothekswissenschaftliche sowie für diesen Bereich interessante
Literatur zu geben. Enthalten sind Beiträge, die der LIBREAS-Redaktion
oder anderen Beitragenden als relevant erschienen.

Themenvielfalt sowie ein Nebeneinander von wissenschaftlichen und
nicht-wissenschaftlichen Ansätzen wird angestrebt und auch in der Form
sollen traditionelle Publikationen ebenso erwähnt werden wie
Blogbeiträge oder Videos beziehungsweise TV-Beiträge.

Gerne gesehen sind Hinweise auf erschienene Literatur oder Beiträge in
anderen Formaten. Diese bitte an die Redaktion richten. (Siehe
\href{http://libreas.eu/about/}{Impressum}, Mailkontakt für diese
Kolumne ist
\href{mailto:zeitschriftenschau@libreas.eu}{\nolinkurl{zeitschriftenschau@libreas.eu}}.)
Die Koordination der Kolumne liegt bei Karsten Schuldt, verantwortlich
für die Inhalte sind die jeweiligen Beitragenden. Die Kolumne
unterstützt den Vereinszweck des LIBREAS-Vereins zur Förderung der
bibliotheks- und informationswissenschaftlichen Kommunikation.

LIBREAS liest gern und viel Open-Access-Veröffentlichungen. Wenn sich
Beiträge dennoch hinter eine Bezahlschranke verbergen, werden diese
durch \enquote{{[}Paywall{]}} gekennzeichnet. Zwar macht das Plugin
\href{http://unpaywall.org/}{Unpaywall} das Finden von legalen
Open-Access-Versionen sehr viel einfacher. Als Service an der
Leserschaft verlinken wir OA-Versionen, die wir vorab finden konnten,
jedoch auch direkt. Für alle Beiträge, die dann immer noch nicht frei
zugänglich sind, empfiehlt die Redaktion Werkzeuge wie den
\href{https://openaccessbutton.org/}{Open Access Button} oder
\href{https://core.ac.uk/services/discovery/}{CORE} zu nutzen oder auf
Twitter mit
\href{https://twitter.com/hashtag/icanhazpdf?src=hash}{\#icanhazpdf} um
Hilfe bei der legalen Dokumentenbeschaffung zu bitten.

Die bibliographischen Daten der besprochenen Beiträge aller Ausgaben
dieser Kolumne finden sich in der öffentlich zugänglichen Zotero-Gruppe:
\url{https://www.zotero.org/groups/4620604/libreas_dldl/library}.

\hypertarget{artikel-und-zeitschriftenausgaben}{%
\section{2. Artikel und
Zeitschriftenausgaben}\label{artikel-und-zeitschriftenausgaben}}

\hypertarget{vermischte-themen}{%
\subsection{2.1 Vermischte Themen}\label{vermischte-themen}}

Laskowska, Aneta (2022). \emph{Publication patterns of academic
librarians from Norwegian higher education institutions 2016-2020.} In:
New Review of Academic Librarianship {[}Latest Articles{]},
\url{https://doi.org/10.1080/13614533.2022.2138478}

Diese Studie zu den wissenschaftlichen Publikationen von norwegischen
Bibliothekar*innen (an wissenschaftlichen Einrichtungen) ist vor allem
deshalb interessant, weil hier einmal tatsächlich Daten aus einem
Forschungsinformationssystem genutzt werden. In Norwegen gibt es seit
Längerem die Anforderung an alle Angestellten an öffentlichen
wissenschaftlichen Einrichtungen -- und damit auch den dort tätigen
Bibliothekar*innen -- ihre Publikationen in einem solchen System,
welches vom Forschungsministerium betrieben wird, einzutragen.
Gleichzeitig gibt es eine Liste von Zeitschriften und Verlagen, die als
wissenschaftlich angesehen werden. Für Publikationen in oder bei diesen
werden jährlich Punkte vergeben. Diese Punkte sind für Forschende und
Institutionen relevant, weil sie die Vergabe von Forschungsmitteln und
die individuelle Karriere beeinflussen. Für Bibliothekar*innen gilt dies
nicht unbedingt, auch wenn sie in einigen Einrichtungen zu
Publikationstätigkeiten angeregt werden.

Die Autorin untersuchte nun auf der Basis der Daten aus dem
Forschungsinformationssystem 224 Publikationen (sowohl Artikel als auch
Monographien), die von Bibliothekar*innen aus Norwegen zwischen 2016 und
2020 veröffentlicht wurden. Es zeigt sich dabei, dass einige
Einrichtungen -- vor allem die Universitätsbibliothek in Tromsø --
publikationsstark sind, andere hingegen nicht. Die Bibliothekar*innen
der den Fachhochschulen im DACH-Raum vergleichbaren Colleges
veröffentlichten praktisch nicht. Thematisch wurde vor allem in den
Erziehungswissenschaften und der Bibliothekswissenschaft gearbeitet, die
meisten Artikel stammten von Einzelautor*innen. Englisch überwog als
Sprache, gefolgt von Norwegisch. Erstaunlich ist, dass in all den Jahren
weiterhin nicht alle Publikationen im Open Access erfolgten (trotzdem
sie von Bibliothekar*innen geschrieben wurden und es das Ziel des
norwegischen Forschungsministeriums ist, bis 2024 einen Open
Access-Anteil von 100\,\% zu erreichen). Was die Studie aber auch zeigt,
ist, dass es sehr wohl möglich ist -- so denn die Strukturen und
vielleicht auch die Anreize dafür existieren -- als Bibliothekar*in
wissenschaftlich zu publizieren. (ks)

\begin{center}\rule{0.5\linewidth}{0.5pt}\end{center}

Khan, Aasif Mohammad ; Loan, Fayaz Ahmad (2022). \emph{Exploring the
review of Google Maps to assess the user opinions about public
libraries}. In: Library Management 43 (2022) 8/9: 601--615,
\url{https://doi.org/10.1108/LM-05-2022-0053} {[}Paywall{]}

Das Interessante an dieser Studie ist vor allem, dass es ein
proof-of-concept darstellt, welches relativ einfach reproduziert werden
kann. Die eigentlichen Ergebnisse hingegen sind etwas sehr spezifisch an
die lokalen Gegebenheiten gebunden.

Grundsätzlich wurden hier Bewertungen, die auf Google Maps zu fünf
Öffentlichen Bibliotheken in Delhi und Neu-Delhi hinterlassen wurden,
gesammelt und dann ausgewertet. Es zeigt sich, dass sich aus diesen ein
relativ konsistentes Bild davon erstellen lässt, wie lokale Nutzer*innen
und Besucher*innen aus anderen Regionen die Bibliotheken und ihre
Angebote wahrnahmen. Grundsätzlich wurden sie in diesem Fall positiv
wahrgenommen, ausser dem Personal, dessen Verhalten fast genauso oft
negativ erwähnt wurde wie gelobt. Am positivsten waren die Rückmeldungen
zu den Angeboten für Kinder. (ks)

\begin{center}\rule{0.5\linewidth}{0.5pt}\end{center}

Finlay, Jayne (2022). \emph{Staff perspectives of providing prison
library services in the United Kingdom}. In: Journal of Librarianship
and Information Science {[}OnlineFirst{]},
\url{https://doi.org/10.1177/09610006221133834}

Ziel des Textes ist es, die Stimmen des Bibliothekspersonals in
Gefängnisbibliotheken in Grossbritannien hörbarer zu machen. Basis dazu
ist eine Umfrage unter diesem Personal sowie zehn vertiefende Interviews
mit Teilnehmenden der Umfrage. Durchgeführt wurde beides für eine
Promotion. Interessant ist dabei, dass zumindest die Interviewten auch
alle eine bibliothekarische Ausbildung haben (das kann man für
Gefängnisbibliotheken in anderen Ländern nicht voraussetzen, wie in
anderen Studien, die schon in dieser Kolumne besprochen wurden, sichtbar
wurde). Zudem ist es in Grossbritannien gesetzlich vorgeschrieben, dass
ein Gefängnis auch eine Bibliothek anbieten muss.

Sichtbar wird, dass das Personal in den Bibliotheken ihre Arbeit
grundsätzlich positiv sieht und auch davon ausgeht, dass diese einen
positiven Einfluss auf das Leben der Gefangenen hat -- immer mit
Einschränkungen und schwer zu messen, aber doch sichtbar. Von vielen
Gefangenen würden die Gefängnisbibliothekar*innen als besondere
Angestellte ohne Uniform wahrgenommen, denen mehr vertraut werden könne
als anderen. Zudem gäbe es immer Gefangene, welche die Bibliothek und
ihre Angebote nutzen würden. Dennoch sieht das Personal auch grosse
Hürden. Sie seien einigermassen isoliert, sowohl vom restlichen
Gefängnisleben -- auch wenn es dabei immer Ausnahmen gibt -- als auch
von anderen Bibliothekar*innen und der gesamten Profession. (Eine Anzahl
der Befragten berichtet aber auch, dass sie aktiv mit der nächsten
Öffentlichen Bibliothek zusammenarbeiten.) Oft würden sie relativ allein
gelassen. In einigen Fällen allerdings würde sich die Gefängnisleitung
sehr für die Bibliothek engagieren oder aber die Bibliothek wäre
explizit in die Bildungsaktivitäten der Anstalt und deren Planungen
einbezogen. Alles in allem gibt der Text einen interessanten Einblick in
die Arbeit der Gefängnisbibliotheken in Grossbritannien, die besser sein
könnte, aber doch auch viele positive Seiten zu haben scheint. (ks)

\begin{center}\rule{0.5\linewidth}{0.5pt}\end{center}

Zellhöfer, David (2022). \emph{Warum stritten wir je um Discovery und
OPAC?: \enquote{Boutique}-Recommender-Systeme als aktuelles
Aufgabengebiet für Digitale Bibliotheken}. In: b.i.t.Online 25 (2022) 3:
233--240,
\url{https://www.b-i-t-online.de/heft/2022-03-fachbeitrag-zellhoefer.pdf}
{[}Freier Zugang, ohne freie Lizenz{]}

Der reisserische Titel dieses Artikels täuscht über die relevanten
Aussagen hinweg. Die Auseinandersetzung um OPAC oder Discovery-System
werden vom Autor nur kurz am Anfang erwähnt, dann aber mit der
Feststellung, dass sich heute Discovery-Systeme durchgesetzt haben,
fallen gelassen.

Relevanter ist die im Artikel vorgenommene zusammenfassende Kritik der
vorhandenen Dis\-covery-Systeme. Diese würden praktisch alle auf eine
veraltete Suchtechnologie setzen, die nicht die Möglichkeiten
semantischer Daten und semantischer Suchen nutzt. Dieses Problem, so der
Autor, sei nicht zu lösen, indem Bibliotheken noch mehr Studien zur
Usability von Discovery-Systemen durchführen -- diese gäbe es schon in
ausreichender Zahl. Aber da die Discovery-Systeme nicht von
Bibliotheken, sondern von Software-Anbietern entwickelt würden, seien
die Einflussmöglichkeiten von Bibliotheken beschränkt. Grundsätzlich
könnten sie Anbieter und Systeme wechseln, aber es gäbe nur wenige
Alternativen. Stattdessen schlägt der Autor vor, dass sich Bibliotheken
darauf konzentrieren sollten, Recommender Systeme zu entwickeln, die
neben den Daten aus den Discovery-Systemen auch semantische Technologien
einsetzen, um den Nutzer*innen kontextualisierte Ergebnisse zu liefern.
(ks)

\begin{center}\rule{0.5\linewidth}{0.5pt}\end{center}

Centerwall, Ulrika (2022). \emph{In plain sight: School librarian
practices within infrastructures for learning}. In: Journal of
Librarianship and Information Science {[}OnlineFirst{]},
\url{https://doi.org/10.1177/09610006221140881}

Fokus des Artikels sind Schulbibliothekar*innen in Schweden und wie
diese ihre eigene Arbeit beschreiben. Dazu wurden 22 Interviews in 14
Schulen geführt, deren Bibliotheken in den vergangenen Jahren von der
Bibliotheksgewerkschaft als \enquote{best practice} ausgezeichnet
wurden. Obgleich Schweden als eines der \enquote{bibliothekarischen
Vorzeigeländer} gilt, es zudem seit 2010 ein Bildungsgesetz gibt,
welches den Zugang zu einer Schulbibliothek für alle Schüler*innen des
Landes vorschreibt und die untersuchten Bibliotheken für ihre
Ausstattung und Arbeit ausgezeichnet wurden, zeigt der Text, dass die
Situation der Schulbibliotheken nicht vollständig positiv ist. Erwähnt
wird zum Beispiel, dass trotz des Gesetzes nur die Hälfte der
Schüler*innen Zugang zu einer Bibliothek hat.

Der eigentliche Schwerpunkt des Textes ist die Eigenwahrnehmung der
Arbeit der Schulbibliotheken. Diese ist laut Auswertung der Interviews
davon gekennzeichnet, dass alle Schulbibliothekar*innen kontinuierlich
daran arbeiten müssen, ihre Arbeit in der Schule bekannt zu machen, ihre
Position zu etablieren und dafür zu werben, dass die Bibliothek einen
Beitrag zu guter Bildung spielt. Die Autorin zeigt, dass laut der
Forschung zu Schulbibliotheken (wobei sie allerdings praktisch nur
US-amerikanische Forschung anführt) empirisch klar ein Zusammenhang
zwischen gut ausgestatteten Schulbibliotheken und besseren Lernerfolgen
von Schüler*innen besteht. Dies würde sich aber nicht im Schulalltag
auszahlen. Vielmehr sei die Arbeit von Schulbibliotheken -- und das dann
wohl, auch wenn die Autorin selbst diesen Schluss nicht zieht, auch in
Zukunft immer wieder neu -- davon geprägt, dass sie um ihren Platz im
Schulalltag kämpfen und Kollaborationen innerhalb der Schulen immer
wieder neu aufbauen müsste.

Obgleich die Studie im schwedischen Kontext verortet ist, lassen sich
auch für den DACH-Raum einige grundsätzliche Erkenntnisse ableiten. Dazu
zählt insbesondere, dass gesetzliche Regelungen für das Vorhandensein
von Bibliotheken noch nicht garantieren, dass diese tatsächlich
eingerichtet und unterhalten werden und, dass Schulbibliotheken offenbar
immer um ihre Rolle und Position in der jeweiligen Schule
\enquote{kämpfen} müssen. (ks)

\begin{center}\rule{0.5\linewidth}{0.5pt}\end{center}

Gonnelli, Elena (2022). \emph{Per le scienze e per diletto: fonti
archivistiche per lo studio delle biblioteche termali}. In:
Bibliothecae.it 11 (2022) 2: 161--193,
\url{https://doi.org/10.6092/issn.2283-9364/16241}

Thermalbäder haben -- nicht nur in Italien, um das es in diesem Artikel
geht -- eine lange Geschichte als Orte für Erholung, Gesundheitspflege,
Freizeitgestaltung und als sozialer Raum. Die Autorin verweist zum
Beispiel kurz auf die römische Antike und die dortigen öffentlichen
Bäder. Im 19. Jahrhundert nahmen sie aber, zusammen mit dem Aufschwung
von Massentourismus, moderner Gesellschaft und Wissenschaft, eine neue
Rolle ein. Zum Beispiel wurden die Wirkungen von Bädern und Bäderkuren
mit wissenschaftlichen Methoden untersucht oder als
\enquote{wissenschaftlich fundiert} angepriesen (mit manchmal
zweifelhafter Evidenz). In dieser Zeit entstanden Bäder als
breitenwirksame Bade- und Kurorte, die teilweise bis heute Bestand
haben.

Die Autorin geht in ihrem Text nun auf Bibliotheken ein, die für den
Betrieb solcher Bäder aufgebaut wurden. Dabei geht es,
erstaunlicherweise, nicht um die \enquote{Freizeitliteratur}, welche im
Rahmen von Bäderbesuchen genutzt wurde, obgleich sie im Text auch kurz
angesprochen werden. Vielmehr thematisiert die Autorin
wissenschaftlich-praktische Spezialbibliotheken, welche vom technischen
und pflegerisch-medizinisch tätigen Personal aufgebaut wurden, um zum
Beispiel den Betrieb der technischen Anlagen oder den Ablauf der
Bäderkuren zu organisieren. Neben einem grundsätzlichen Überblick geht
sie auf ein explizites Beispiel ein -- die Bibliothek der
Montecatini-Terme in der gleichnamigen Stadt in der Toskana. Diese hat
sich, inklusive handgeschriebenem Zettelkatalog, Möbeln und zahlreichen
Unterlagen, in Stadtbibliothek, Stadtarchiv und den Räumen der
Verwaltung der Terme bis heute erhalten, obwohl sie nicht mehr aktiv
betrieben wird. Im Artikel wird auf den inhaltlichen Aufbau der
Bibliothek -- vor allem technische Werke -- eingegangen, auf die
Geschichte der Bibliothek sowie auf Nutzungsbedingungen. Zudem wird eine
Anzahl von Bildern der Möbel und des Katalogs geliefert.

Die Autorin betont zu Recht, dass diese konkrete Bibliothek vielleicht
deshalb heraussticht, weil sie heute noch so gut und vollständig
erhalten ist (im Zustand der 1960er-Jahre), aber dass es in den
zahlreichen anderen Thermalbädern, die im 19. und frühen 20. Jahrhundert
florierten, ähnliche dieser Spezialbibliotheken gegeben haben muss, die
zu untersuchen die Aufgabe weiterer Studien wäre. (ks)

\begin{center}\rule{0.5\linewidth}{0.5pt}\end{center}

Rösch, Hermann; Sondermann, Frieder (2023): \emph{\enquote{Starb den
schönen Tod in seinem Berufe}: Ein neu entdecktes zeitgenössisches
Dokument zu Friedrich Adolf Eberts Sturz von der Bücherleiter}. In:
O-Bib. Das offene Bibliotheksjournal 10 (1), S. 1--18,
\url{https://doi.org/10.5282/o-bib/5908}.

Friedrich Adolf Ebert (1791--1834), seiner Zeit Oberbibliothekar in
Dresden, ist heute als einer der Begründer der Bibliothekswissenschaft
und insbesondere für seine Todesursache bekannt: Er starb nach einem
Sturz von der Bibliotheksleiter. Die Autoren beleuchten nun diese
Todesumstände etwas genauer, nachdem in einem Brief des Gelehrten Carl
August Böttiger (1760--1835) an den Oberbibliothekar der Herzoglichen
öffentlichen Bibliothek zu Gotha, Friedrich Jacobs (1764--1847), neue
Informationen zu Eberts Lebensumständen und dem berühmten Leitersturz
entdeckt wurden. Dass Ebert einen recht streitbaren und eigensinnigen
Charakter besaß, war schon aus anderen Quellen bekannt. Böttiger (der
mit Ebert schon seit einiger Zeit zerstritten war) geht in seinem Brief
noch einen Schritt weiter und beschreibt ihn als arbeitsfaulen und
prokrastinierenden Alkoholiker. (eb)

\begin{center}\rule{0.5\linewidth}{0.5pt}\end{center}

Bolick, Josh ; Bonn, Maria ; Cross, Will (Hrsg.) {[}in Vorbereitung für
Ende 2023{]}: \emph{Scholarly Communication Librarianship and Open
Knowledge}. \url{https://lisoer.wordpress.ncsu.edu/book/}

Hollister, Christopher V. ; Jensen, Jennifer M. K. (2023):
\emph{Research Productivity Among Scholarly Communication Librarians}.
In: Journal of Librarianship and Scholarly Communication 11.1,
\url{https://doi.org/10.31274/jlsc.15621}

Robrecht, Michèle (2022): \emph{Positionsbestimmung wissenschaftlicher
Bibliotheken in der externen Wissenschaftskommunikation am Beispiel des
Fraunhofer-Fachinformationsmanagements}. Masterarbeit TH Köln,
\url{https://doi.org/10.24406/publica-303}

Das Thema Wissen(schaft)skommunikation wird auch im Bibliothekswesen
immer relevanter, wie zum Beispiel Robrecht 2022 aufzeigt. Die
amerikanischen Kolleg:innen Christopher Hollister und Jennifer Jensen
wollten herausfinden, in welchem Umfang Bibliothekar:innen, zu deren
Aufgabenbereich \enquote{scholarly communication responsibilities}
gehören, selbst Forschung und Lehre betreiben, welche Motivation
dahintersteht und wie sich das auf ihre anderen Tätigkeiten auswirkt.
Ein Ergebnis ihrer Umfrage ist: Viele Teilnehmer:innen gaben an, dass
Scholarly Communication kein Thema in ihrer Ausbildung war und sie daher
zum Teil unter einer Art Impostor-Syndrom leiden, auch wenn sie schon
seit Jahren in diesem Bereich arbeiten. Daraus ergeben sich Überlegungen
für die zukünftige Gestaltung von Ausbildungsprogrammen und
Stellenbeschreibungen beziehungsweise -anforderungen. In diesem Kontext
ist auch das im Artikel erwähnte Handbuch \enquote{OER + ScholComm} von
Bolick/Bonn/Cross interessant, das Ende 2023 erscheinen soll. (vv)

\begin{center}\rule{0.5\linewidth}{0.5pt}\end{center}

Collinson, Timothy ; Porter, Hannah ; Work, Collin K. (2021):
\emph{University of Portsmouth library subject pages: a flexible
in-house system for guidance and resource discovery}. In: The Journal of
Academic Librarianship 2022 48.1:102453
\url{https://doi.org/10.1016/j.acalib.2021.102453} {[}Paywall{]}

Der Aufbau und vor allem die Pflege von Fachinformationsseiten --
\enquote{Informationsquellen für die Anglistik},
\enquote{Recherche-Tipps für Biolog:innen} und ähnliche -- sind
bekanntlich sehr aufwendig. Ein limitierender Faktor kann dabei sein,
dass die Fachreferent:innen die Seiten nicht selbst bearbeiten können
und Änderungswünsche an die Kolleg:innen der Webadministration melden
müssen. An der Bibliothek der University of Portsmouth wurde ein
datenbankbasiertes modulares System entwickelt, das ein kollaboratives
Arbeiten ermöglicht, ohne sich in das Content-Management-System
einarbeiten zu müssen, mit dem die übrigen Bibliotheksseiten gepflegt
werden. Die Fachinformationsseiten sollen sich dennoch nahtlos in den
Webauftritt einfügen: sie sollten ähnlich strukturiert sein, aber
dennoch fachspezifische Inhalte bieten können. Ein Beispiel für eine mit
dem neuen System erstellte Seite:
\url{https://library.port.ac.uk/subject/sub37.html}. Für Bibliotheken,
die über das \enquote{Handling} ihrer Fachinformationsseiten nachdenken,
könnte die \enquote{Portsmouther Variante}, die der Artikel kompakt
vorstellt, einige Anregungen geben. (vv)

\hypertarget{covid-und-die-bibliotheken}{%
\subsection{2.2 Covid und die
Bibliotheken}\label{covid-und-die-bibliotheken}}

Garner, Jane ; Wakeling, Simon ; Hider, Philip ; Jamali, Hamid R. ;
Kennan, Mary Anne ; Mansourian, Yazdan ; Randell-Moon, Holly (2022).
\emph{The lived experience of Australian public library staff during the
COVID-19 library closures}. In: Library Management 43 (2022) 6/7:
427--438, \url{https://doi.org/10.1108/LM-04-2022-0028} {[}Paywall{]}

In der vorgestellten Interviewstudie geht es -- wie im Titel ersichtlich
-- darum, wie Bibliothekar*innen in Australien die Schliessungen ihrer
Bibliotheken während 2020 und 2021 erlebt haben. Dabei befragten die
Autor*innen Personen aus drei Bibliotheken, welche drei verschiedene
geographische Kontexte (eine Grossstadt, eine Mittelstadt und eine
«remote location») abdeckten. Grundsätzlich belasteten die Schliessung
und die gesamte Situation die Bibliothekar*innen alle, wobei einige dies
positiv verarbeiteten und andere eher negativ. Der Wechsel -- für viele
-- in das Arbeiten von zuhause wurde grundsätzlich gut bewältigt, war
aber nicht ohne Herausforderungen. Auch fiel dies einigen schwer.
Gleichzeitig gab es unterschiedliche Einschätzungen von denen, welche
trotzdem dazu eingeteilt wurden, weiterhin physisch vor Ort zu arbeiten,
ob sie dies als geringe Wertschätzung ihrer selbst oder als Möglichkeit
für persönliche Kontakte interpretieren sollten. Als Besonderheit heben
die Autor*innen hervor, dass Bibliothekar*innen ihre Aufgabe darin
sahen, ihre jeweilige Community während der Krise zu unterstützen,
während sie diese gleichzeitig selber durchlebten. (ks)

\begin{center}\rule{0.5\linewidth}{0.5pt}\end{center}

Nash, Maryellen ; Lewis, Barbara ; Szempruch, Jessica ; Jacobs,
Stephanie ; Silver, Susan (2022). \emph{Together, Apart: Communication
Dynamics among Academic Librarians during the COVID-19 Pandemic}. In:
College \& Research Libraries 83 (2022) 6: 946--965,
\url{https://doi.org/10.5860/crl.83.6.946}

Mittels Ende 2020 durchgeführter Umfragen prüft die Studie die Annahme,
dass sich durch die Online-Arbeit während des erstens Jahres der Covid
19-Pandemie das Zusammengehörigkeitsgefühl von Teams, die an
Wissenschaftlichen Bibliotheken (in den USA) arbeiten, verbessert hätte.
Zu dieser Annahme kamen die Autor*innen, weil sie es selber in ihrem
Team erlebt hatten. Die Ergebnisse zeigen hingegen, dass es zwar kleine
Veränderungen gab, aber dass sich die Einschätzungen der an der Umfrage
teilnehmenden Bibliothekar*innen (N = 299) praktisch die Waage hielten:
Ungefähr gleich viele fanden, dass sie mehr, gleich viel oder weniger
Kontakt mit ihrem jeweiligen Team hatten. Zudem schätzte die
überwiegende Anzahl schon Ende 2020 ein, dass sich die Veränderungen
durch die Telearbeit nicht über die Pandemie hinaus verstetigen würden.
(ks)

\begin{center}\rule{0.5\linewidth}{0.5pt}\end{center}

\pagebreak

Lantzy, Tricia (2022). \emph{Involuntary Online Learners and the
Library: How the Pandemic Closures Affected College Students' Library
Research}. In. Journal of Library \& Information Services in Distance
Learning, \url{https://doi.org/10.1080/1533290X.2022.2149662}
{[}Paywall{]}

Eine Umfrage unter Studierenden der California State University San
Marcos, welche 2020 / 2021 Einführungskurse besuchten, die von
Bibliothekar*innen gehalten wurden und in die Recherche und
Bibliotheksnutzung einführten, sollte Aufschluss darüber geben, wie
deren Recherchepraxen und Sicht auf die Bibliothek sich während der
Covid 19-Pandemie veränderten haben. Die Population war relativ klein
(255 Studierende hätten antworten können, 134 Antworten gab es), aber
innerhalb dieser gab es zwei Tendenzen, die im Artikel durch weitere
Daten der Bibliothek ergänzt wurden. Erstens fanden ungefähr gleich
viele Studierende die Situation belastend oder gerade nicht belastend.
Es gab also keine allgemein geteilte Ansicht, sondern offenbar
verschiedene Verarbeitungstendenzen. Zudem hatten fast alle Studierenden
den Eindruck, dass sie trotz der Pandemiesituation die Recherche für
ihre Studierendenarbeiten durchführen konnten, wenn es auch eine Anzahl
von ihnen gab, welche die Arbeit im physischen Raum Bibliothek
vermissten. Zweitens gab es eine Tendenz der Studierenden, den direkten
Kontakt mit Bibliothekar*innen zu vermeiden, auch bei Nachfragen. Zwar
bot die Bibliothek Chats und Videocalls, aber sie wurden viel weniger
genutzt, als dies bei direkten persönlichen Kontakten vor der Pandemie
üblich war. (Dies hat sich, laut Artikel, auch nach 2021 nicht
geändert.) Studierende informierten sich eher über andere Wege, was die
Autorin des Textes zu der Einschätzung bringt, dass Bibliotheken jetzt
mehr Ressourcen in das Erstellen und Pflegen von Online-Seiten
investieren müssten. (ks)

\begin{center}\rule{0.5\linewidth}{0.5pt}\end{center}

Dalmer, Nicole K. ; Sawchuk, Dana ; Ly, Mina (2022). \emph{\enquote{I
felt there was a big chunk taken out of my life}: COVID-19 and older
adults' library-based magazine leisure reading.} In: Leisure Studies,
\url{https://doi.org/10.1080/02614367.2022.2148719} {[}Paywall{]}

Fachlich kommen die Autor*innen dieser Studie aus der Leseforschung,
nicht der Bibliothekswissenschaft. Die 21 Interviews, welche sie in
diesem Text auswerteten, wurden zwar explizit zu der Frage geführt, wie
ältere Personen (56 bis 81 Jahre) in Ontario, Kanada, die normalerweise
in Öffentlichen Bibliotheken Zeitschriften lesen, dies während der
Schliessungen während der Covid-19 Pandemie taten. Aber eingebettet sind
sie in eine grössere Studie zum Zeitschriftenlesen. Die Fragen, die
gestellt wurden, und die Forschung, in die sie eingebettet werden,
stammen aber fast durchgängig aus der Leseforschung. Es geht also zum
Beispiel darum, wozu Menschen Zeitschriften lesen und weniger darum, wie
sie Bibliotheken an sich wahrnehmen -- aber in diesem Text immer unter
dem Fokus, warum sie dies in Bibliotheken taten.

Was sich zeigt, ist: (a) Das Lesen von Zeitschriften in Bibliotheken ist
eingebunden in andere Tätigkeiten. Es geht den Interviewten oft um die
Gestaltung ihrer Freizeit, zu der Besuche in der Bibliothek gehören. (b)
Die meisten lesen Zeitschriften zur Unterhaltung, einige aber auch, um
andere ihrer Hobbys zu unterstützen (also zum Beispiel als Gärtner*in
Zeitschriften über das Gärtnern zu lesen). (c) Der Raum Bibliothek
(inklusive des sozialen Settings) hat eine Bedeutung. Es geht explizit
darum, dort, zwischen anderen Menschen, zu lesen. (d) Für viele ist es
auch wichtig, dass die Zeitschriften kostenlos zugänglich sind und es
gleichzeitig eine Auswahl von ihnen gibt. Selber kaufen sie kaum
Zeitschriften. (e) Die Einschränkungen während der Schliessungen führten
dazu, dass viele das Lesen von Zeitschriften ganz aufgaben. Sie
ersetzten sie nicht mit anderen Medien oder mit digitalen Angeboten. (f)
Eine kleine Anzahl der Interviewten stieg auf digitale Zeitschriften um,
auch über digitale Angebote ihrer Bibliotheken selber. Ob dies
langfristig so bleiben wird oder sie mit der Zeit wieder auf gedruckte
Zeitschriften zurückgreifen werden, ist offen. (g) Grundsätzlich zeigte
sich, dass die älteren Leser*innen von Zeitschriften eine sehr
heterogene Gruppe sind, deren Lesemotivationen und -gewohnheiten schwer
zu gruppieren sind. (ks)

\begin{center}\rule{0.5\linewidth}{0.5pt}\end{center}

Kosciejew, Marc (2021). \emph{The coronavirus pandemic, libraries and
information: a thematic analysis of initial international responses to
COVID-19}. In: Global Knowledge, Memory and Communication 70 (2021) 5/5:
304--324, \url{http://dx.doi.org/10.1108/GKMC-04-2020-0041}
{[}Paywall{]}

Kosciejew, Marc (2022). \emph{National archives, records and the
coronavirus pandemic: a comparative thematic analysis of initial
international responses to COVID-19}. In: Global Knowledge, Memory and
Communication 71 (2022) 8/9: 732--753,
\url{http://dx.doi.org/10.1108/GKMC-04-2021-0066} {[}Paywall{]}

Der Autor untersucht in diesen beiden Texten jeweils mit der gleichen
Methodik und Fragestellung, welche Themen -- im Artikel von 2021 --
einige Bibliotheksverbände beziehungsweise -- im Artikel von 2022 --
einige Nationalarchive Anfang 2020 in ihren jeweils ersten Erklärungen
zur gerade beginnenden Covid-19-Pandemie behandelten. Der zweite Text
stellt eine explizite Verbindung zwischen Bibliotheken und Archiven als
\enquote{cultural memory institutions} her. In diesem Zusammenhang
erwähnt er auch Museen, insoweit ist wohl ein weiterer Text zu diesen zu
erwarten.

Grundsätzlich nahm der Autor die Erklärungen von Bibliotheksverbänden
und Nationalarchiven aus einigen englischsprachigen Ländern des Globalen
Nordens und analysierte sie jeweils auf vorkommende Themen. Dabei zeigte
sich in beiden Fällen, dass es Gemeinsamkeiten, aber auch nationale
Eigenheiten gab. Ausserdem gab es Gemeinsamkeiten zwischen
Bibliotheksverbänden und Nationalarchiven. Immer ging es um die Themen
Schliessung von Einrichtungen, den möglichen Weiterbetrieb von Services
von Bibliotheken oder Archiven sowie den möglichen \enquote{remote
access} zu Medien beziehungsweise Archivalien. Die Ergebnisse deuten
darauf hin, dass Bibliotheken und Archive zumindest in
englischsprachigen Ländern des Globalen Nordens jeweils auf die
Herausforderungen am Anfang der Covid-19-Pandemie ähnlich reagierten.
(ks)

\begin{center}\rule{0.5\linewidth}{0.5pt}\end{center}

Pryce, Trecia Latoya ; Russell, Jollette ; Crawford, Marsha Nicola ;
McDermott, Joan Opal ; Christina, Ariel ; Perkins, Nordia (2022).
\emph{Experiences, perspectives, and emerging frameworks: COLINET
libraries response to the COVID-19 pandemic}. In: Global Knowledge,
Memory and Communication 71 (2022) 8/9: 754--771,
\url{https://doi.org/10.1108/GKMC-03-2021-0055} {[}Paywall{]}

Gestützt auf Interviews, eine Fokusgruppe und eine Umfrage unter
Bibliothekar*innen des Netzwerks der meisten jamaikanischen
Hochschulbibliotheken wird in diesem Artikel dargestellt, wie diese
Bibliotheken auf die Herausforderungen während der Covid-19 Pandemie
reagierten. Die Autor*innen betonen zu Beginn, dass die Pandemie für
Länder des Globalen Südens, wie Jamaika, Herausforderungen darstellen,
die noch grösser waren als im Globalen Norden. In den Ergebnissen zeigt
sich das aber kaum: Die Bibliotheken an jamaikanischen Hochschulen
reagierten ähnlich, wie dies auch aus Texten über Bibliotheken im
Globalen Norden (die ja unter anderem in dieser Kolumne referiert
werden) bekannt ist. Sie alle setzten schnell auf elektronische
Angebote, digitale Medien und flexible Formen des Arbeitens. Einige
Bibliotheken schienen recht gut auf eine Krise vorbereitet gewesen zu
sein, andere weniger. Einige mussten Personal entlassen, aber am Ende
fanden die meisten Bibliotheken durch die Pandemie zu neuen Aufgaben und
Arbeitsstrukturen. (ks)

\begin{center}\rule{0.5\linewidth}{0.5pt}\end{center}

Reid, Peter H. ; Mesjar, Lyndsay (2023). \emph{\enquote{Bloody amazing
really}: voices from Scotland's public libraries in lockdown}. In:
Journal of Documentation 79 (2023) 2: 301--319,
\url{https://doi.org/10.1108/JD-03-2022-0067} {[}Paywall{]}
{[}OA-Version:
\url{https://rgu-repository.worktribe.com/output/1681684}{]}

Der Text ist eine weitere \enquote{explorative Studie} -- also letztlich
eine Sammlung von semistrukturierten Interviews, deren Ergebnisse
zusammengefasst werden -- zu den Erfahrungen von Bibliotheken während
der Covid-19-Pandemie. Fokus sind hier Öffentliche Bibliotheken in
Schottland, befragt wurden 15 Leitungen auf Ebene der Gemeinden. Mehr
Interviews wären möglich gewesen (bei 32 solcher Leitungen im Land),
aber die Ergebnisse schienen den Autor*innen nach den geführten
gesättigt. Es geht dabei um den gesamten frühen Verlauf der Pandemie,
also den ersten Lockdowns bis zum Aufbau von Lieferdiensten und den
ersten Öffnungen nach einigen Monaten. Dabei wird ein sehr positives
Bild der Bibliotheken, des Personals und auch der zukünftigen Potenziale
von Bibliotheken als Orte gezeichnet. Nach einigen Wochen der
Orientierung hätten sie sehr schnell eigene Rollen gefunden und ihre
Arbeit den Herausforderungen entsprechend organisiert. Teilweise scheint
die Darstellung etwas übertrieben positiv -- leicht kommt die Frage auf,
warum Bibliotheken sich überhaupt Gedanken über ihre Zukunft machen,
wenn sie in Krisen so schnell, flexibel und auf die Nutzer*innen
ausgerichtet reagieren können. (ks)

\begin{center}\rule{0.5\linewidth}{0.5pt}\end{center}

Ragon, Bart ; Whipple, Elizabeth ; Rethlefsen, Melissa L. (2022).
\emph{Except for my commute, everything is the same: the shared lived
experience of health sciences libraries during the COVID-19 pandemic}.
In: Journal of the Medical Library Association 110 (2022) 4: 419--428,
\url{https://doi.org/10.5195/jmla.2022.1475}

Auch diese Studie fragt nach den Erfahrungen von Bibliothekar*innen
während der Frühphase der Covid-19-Pandemie, in diesem Fall spezifisch
solche an Medizinbibliotheken in den USA und Kanada. Eine Besonderheit
ist, dass insgesamt drei Umfragen unter den gleichen Bibliothekar*innen
durchgeführt wurden, und zwar im April 2020, August 2020 und Februar
2021. So lassen sich Veränderungen nachvollziehen.

Wie auch in zahlreichen vergleichbaren Studien, die in letzter Zeit
erschienen sind, zeigt sich, dass die Bibliotheken, an denen die
befragten Bibliothekar*innen angestellt sind, relativ schnell und
flexibel auf die sich bietenden Herausforderungen reagierten. Da
Medizinbibliotheken an sich schon viele digital basierte Angebote machen
und digitale Arbeitsweisen etabliert haben, war dies vielleicht sogar
noch einfacher als bei anderen Bibliothekstypen. Bemerkenswert ist an
den Ergebnissen der Studie, dass sie eine Veränderung in der
Grundstimmung der Bibliothekar*innen zeigen: War die Stimmung zuerst
recht positiv (in dem Sinne, dass ein \enquote{neuer Teamgeist} gesehen
und eine wachsende Bedeutung der Bibliotheken erwartet wurde), zeigte
sich im August 2020 eher Erschöpfung und ein relativ negativer Ausblick.
Im Februar 2021 hatte sich dies in gewisser Weise
\enquote{eingependelt}. Die negativen und positiven Äusserungen und
Erwartungen hielten sich mehr oder minder die Wage. Für eine weitere
\enquote{Aufarbeitung} der bibliothekarischen Arbeit während dieser
Jahre wird also wohl zu beachten sein, dass sich die Haltung der
Bibliothekar*innen selbst über die Zeit veränderte. (ks)

\begin{center}\rule{0.5\linewidth}{0.5pt}\end{center}

Ma, Jinxuan ; Wang, Ting ; Lund, Brady (2023). \emph{Analyzing Public
Libraries as Civic Agents in Advocating for COVID-19 Vaccine Uptake}.
In: Public Library Quarterly {[}Latest Articles{]},
\url{https://doi.org/10.1080/01616846.2023.2197842} {[}Paywall{]}

Auf der Basis der Informationen, die auf den Homepages von 80 zufällig
ausgewählten Public Libraries aus den USA verfügbaren waren, versuchten
die Autor*innen zu klären, wie diese dazu beitrugen, Impfungen gegen
Covid-19 zu verbreiten. (Benutzt wird dabei auch GPT-3, um diese
Informationen zu clustern, was nicht sehr sinnvoll erscheint, da es sich
nicht um so viele Daten handelt. Aber es zeigt, dass es möglich wäre,
auch mehr solcher Daten zu clustern.) Gezählt werden dabei alle
möglichen Aktivitäten von Bibliotheken, von ausgelegten Flyern über
Informationsstände bis hin zu Impfungen, welche direkt in den
Bibliotheken (von ausgebildetem Personal, nicht den Bibliothekar*innen
selbst) vorgenommen wurden.

Interessant ist, dass dabei ein Zusammenhang zwischen der Grösse der
Stadt, in denen die Bibliotheken angesiedelt sind, und deren Angeboten
zu sehen ist: Je grösser die Stadt -- und damit wohl auch je mehr
Ressourcen die Bibliothek beziehungsweise das Bibliothekssystem zur
Verfügung hat --, je mehr Informationen wurden von den Bibliotheken
aktiv verbreitet und je mehr konkrete Angebote wurden gemacht. Ansonsten
werten die Autor*innen ihre Ergebnisse so, dass Public Libraries einen
grossen Einfluss auf die Verbreitung der Impfungen gehabt hätten. Das
scheint aber eine übertrieben positive Wertung zu sein, die sich
zumindest mit den Daten selbst nicht begründen lässt. (ks)

\hypertarget{critical-librarianship}{%
\subsection{2.3 Critical Librarianship}\label{critical-librarianship}}

Foster, Elizabeth ; McLaughlin, Anne ; Meyer, Zia ; Nuzum, Derek ;
Rapchak, Marcia ; Reis, Heidi ; Saunders, Jess ; Wiley, Paula (2022).
\emph{"They Don\textquotesingle t Necessarily Play Nice with Power
Structure": Experiences in a Critical Librarianship Reading Group}. In:
Journal of Radical Librarianship 8 (2022): 53--74,
\url{https://journal.radicallibrarianship.org/index.php/journal/article/view/71}

In dieser Studie befragte sich praktisch eine Gruppe von
Bibliothekar*innen, Studierenden im Bibliothekswesen und
Bibliothekswissenschaftler*innen, welche sich monatlich online für eine
Stunde zu einer Lesegruppe zu \enquote{critical librarianship}
zusammenfinden, gegenseitig (plus einige andere Mitglieder der Gruppe)
dazu, wieso sie an diesen Treffen teilnehmen, was sie daraus für ihre
eigene Arbeit oder ihr Studium mitnehmen und welche Probleme sie mit der
Gruppe sehen. Grundsätzlich finden sie das Lesen und gemeinsame
Diskutieren von kritischen Themen hilfreich dafür, in anderen
Zusammenhängen kritisch zu bleiben sowie um Strukturen zu hinterfragen.
Problematisch ist für sie, dass eine reine Lesegruppe, die keine
weiteren politischen Aktionen plant, rein performativ funktionieren
kann: Die Mitglieder diskutieren, aber es folgt erst einmal nichts
daraus. Interessant ist an diesem Text aber vor allem, dass solche
Lesegruppen erfolgreich funktionieren können (und auch schon, was in der
Literaturdiskussion sichtbar wird, in den vergangenen Jahren in anderen
Texten vorgestellt wurden). Im Fazit geben die Autor*innen kurze
Hinweise dazu, was beim Aufbau und Betreiben solcher kritischen
Lesegruppen beachtet werden sollte. (ks)

\begin{center}\rule{0.5\linewidth}{0.5pt}\end{center}

Benoff, Emily (2022). \emph{The Clash of the Commons: An Imagined
Library Commons Discourse.} In: Urban Library Journal 28 (2022) 2:
Article 2, \url{https://academicworks.cuny.edu/ulj/vol28/iss2/2}

Die Diskussion des \enquote{Commons}-Begriffs, seiner Geschichte und
seiner heutigen Umsetzung in Bibliotheken ist nicht nur äusserst
kritisch, sondern auch an die Geschichte des Kolonialismus in den USA
und Kanada gebunden. Diese Gesellschaften werden als \enquote{White
Settler Communities} verstanden und gefragt, wie der Diskurs um
\enquote{Commons}, wie er in britischen Kolonialprojekten auf dem
Kontinent umgesetzt wurde, mit dem heute im bibliothekarischen Diskurs
und der bibliothekarischen Praxis dieser beiden Länder verbreiteten
Verständnis von \enquote{Commons} verbunden ist. Grundsätzlich stellt
die Autorin fest, dass nicht nur die Begriffsgeschichte eine
kolonialistische Vergangenheit hat -- unter der Idee der Commons wurde
die gemeinsame Nutzung von \enquote{leerem Land} durch Siedler*innen
verstanden --, sondern auch heute noch verbunden ist mit der Vorstellung
des \enquote{Schaffens von Räumen}, ohne zu fragen, was in diesen Räumen
schon ist und wer sie nutzt. Das führe dazu, dass unter dem Begriff
\enquote{Commons} Räume und Strukturen geschaffen werden, die bestimmte
Nutzungsweisen und Nutzer*innengruppen implizieren und andere wiederum
ausschliessen. Das Bibliothekswesen würde somit, obgleich die Commons
als offen und zukunftsgewandt verstanden werden, praktisch an der
Fortschreibung von in der kolonialen Geschichte angelegten Ausschlüssen
mitarbeiten, was sich insbesondere an indirekter Mitwirkung an der
Gentrifizierung von Innenstädten zeige. Grundsätzlich fordert die
Autorin, (a) dass der bibliothekarische Diskurs die Geschichte und den
Inhalt der Begriffe, die verwendet werden, mit reflektieren muss, (b)
dass bei der \enquote{Umsetzung} von Konzepten und Begriffen in die
bibliothekarische Praxis auch darauf geachtet werden muss, welche
Konsequenzen dies tatsächlich hat und (c) dass dies vor allem die
Analyse von Machtstrukturen und Verdrängungsprozessen beinhalten muss.

Der Text ist eine tiefgehende Kritik, aber vor allem des US-amerikanisch
/ kanadischen Öffentlichen Bibliothekswesen. Im DACH-Raum ist der
Begriff Commons im Bibliothekswesen nicht so verbreitet (dafür aber der
des \enquote{Dritten Ortes}) und die Geschichte des Kolonialismus ist
eine andere (obgleich auch keine, die \enquote{vorbei} wäre). Insoweit
lässt sich der Beitrag mit grossem Abstand lesen. Aber es wird auch
klar, dass ein solches Nachdenken über die (möglichen) Inhalte von
Begriffen, die im Bibliothekswesen verbreitet sind, und den eventuell
ausschliessenden Ergebnissen, wenn sie in die Praxis übersetzt werden,
hierzulande gänzlich fehlt. (ks)

\begin{center}\rule{0.5\linewidth}{0.5pt}\end{center}

Clark, Jasmine L. ; Lischer-Katz, Zack (2023). \emph{(In)accessibility
and the technocratic library: Addressing institutional failures in
library adoption of emerging technologies}. In: First Monday 28 (2023)
1--2, \url{https://doi.org/10.5210/fm.v28i1.12928}

Im Titel kündigen die Autor*innen an, über Accessibility von
Bibliothekstechnologie zu schreiben. Aber das ist nur ein kleiner Teil
dieses Essays. Vielmehr nutzen sie den Fakt, dass die schnelle, an
Modellen der Softwareentwicklung orientierte Erarbeitung und Integration
von Technologie in Bibliotheken in den vergangenen Jahrzehnten
systematisch Fragen der Accessibility in den Hintergrund rücken würde
(und dafür Ideen wie Innovation um der Innovation willen folgten),
dafür, eine grundsätzliche Kritik an den Formen von Softwareentwicklung
in Bibliotheken zu äussern. Diese würden, so die Autor*innen, den
Zielsetzungen und Denkweisen von Soft- und Hardwareunternehmen sowie
Start-Ups folgen, obgleich diese für Bibliotheken nicht anwendbar seien.
Der Fokus auf technische Lösungen und Innovationen sowie schnelle
Veränderungen hätte dazu geführt, dass Bibliotheken sich als
grundsätzlich unmodern und veränderungswürdig begreifen würden, dass sie
Prämissen setzen würden, die sie von ihren eigentlichen, auf die
Gesellschaft ausgerichteten, Aufgaben entfernen würden und dabei
gleichzeitig bestimmte Gruppen von Personen (gut ausgebildet,
able-bodied, sozial abgesichert) in den Mittelpunkt von
Bibliotheksentwicklung gestellt hätten, während alle anderen Mitglieder
der Gesellschaft, und deren Bedürfnisse, in den Hintergrund gerückt
worden wären. Es ist ein recht wütender Essay, der umgreifend ausholt
und -- wie oft bei solchen Texten -- weniger Lösungen anbietet, als die
(als falsch wahrgenommene) Situation zu beschreiben. (ks)

\begin{center}\rule{0.5\linewidth}{0.5pt}\end{center}

Jimenez, Andrea ; Vannini, Sara ; Cox, Andrew (2022). \emph{A holistic
decolonial lens for library and information studies}. In: Journal of
Documentation 79 (2023) 1: 224--244,
\url{https://doi.org/10.1108/JD-10-2021-0205} {[}Paywall{]}
{[}OA-Version \url{https://eprints.whiterose.ac.uk/187903/}{]}

In einem weiteren Beitrag zu der Frage, wie eine dekolonial orientierte
Bibliotheks- und Informationswissenschaft aussehen und erreicht werden
kann, skizzieren Jimenez et al.~zuerst die Gefahr, \enquote{dekolonial}
in einem \enquote{neoliberalen} (so die Autor*innen) Sinn einfach nur
als Diversifikation der Literaturlisten zu verstehen. Vielmehr hätte der
Begriff dekolonial einen fundamentalen, an den Denkstrukturen von
Gesellschaft und Wissenschaft ansetzenden Ansatz, der erhalten bleiben
müsse, um dessen kritisches Potenzial hin zu einer veränderten,
gerechteren Wissenschaft auszuschöpfen. Dekolonial denken und handeln
hiesse, langfristige Lern- und Veränderungsprozesse anzugehen, die auch
zu grundlegenden institutionellen und persönlichen Transformationen
führen müssten.

Anschliessend nutzen die Autor*innen ein \enquote{framework on
decolonisation} (von Sabelo J. Ndlovu-Gatsheni), um solche Veränderungen
im Bereich Bibliotheks- und Informationswissenschaft anzustossen. Sie
benennen dazu eine, vom Framework vorgegebene, Anzahl von
Themenbereichen, in denen Veränderung notwendig wäre und anschliessend,
offene Fragen und Handlungsbereiche für Forschende in dieser
Wissenschaft. Das Ganze ist, wie vieles, was zum Thema in den letzten
Jahren geschrieben wurde, auf einer recht \enquote{hohen Ebene}
angesiedelt -- in konkretes Handeln muss es weiterhin von einzelnen
Aktiven umgesetzt werden. Etwas ärgerlich ist dies, weil eine Kritik der
Autor*innen selber ist, dass bislang schon vereinzelt in Projekten
gehandelt werden würde, denen eine Verbindung fehlen würde, um einen
nachhaltigen Einfluss zu haben.

Etwas erstaunlich ist zudem, dass bei der Thematisierung der
postkolonialen Strukturen, mit denen die Wissenschaft verbunden ist, im
ersten Teil des Artikels nicht auch thematisiert wird, dass Bibliotheken
als konkrete Institutionen immer direkt mit Trägereinrichtungen
verbunden sind. Das beeinflusst ihre konkreten
Veränderungsmöglichkeiten. Nur, zum Beispiel, wenn sich das gesamte
Universitätssystem dekolonial verändert, scheint auch eine
Dekolonialisierung von Hochschulbibliotheken denkbar. Aber diese
Abhängigkeit wird im Artikel, der den Anspruch erhebt, eine
ganzheitliche Analyse vorzulegen, nicht thematisiert. (ks)

\hypertarget{open-science-und-forschungsdatenmanagement}{%
\subsection{2.4 Open Science und
Forschungsdatenmanagement}\label{open-science-und-forschungsdatenmanagement}}

Petters, Jonathan L. ; Hilal, Amr E. ; Ogier, Andrea L. (2022). \emph{An
Assessment of Research Data Services Through Client Interaction
Records}. In: Journal of Librarianship and Scholarly Communication 10
(2022) 1, \url{https://doi.org/10.31274/jlsc.14439}

Eine Gruppe des Forschungsdatamanagement (FDM)-Teams der Virginia Tech
University Libraries wertet in diesem Text Daten aus, die das Team bei
jeder Interaktion mit Nutzer*innen zwischen 2016 und 2020 erhoben hat.
Die Daten wurden jeweils direkt nach der Interaktion aufgenommen. Im
Text werden sie daraufhin ausgewertet, wie erfolgreich das 2016
etablierte Team agiert hat. Dabei geht die Gruppe davon aus, dass FDM
heute grundsätzlich als Angebot von Wissenschaftlichen Bibliotheken,
zumindest in den USA, etabliert sei.

Was die Daten zeigen, ist, dass die Zahl der Kontakte mit Forschenden
und Studierenden in den ersten Jahren nach der Etablierung anstieg, aber
jetzt auch schon ein gewisses Plateau erreicht hat. 2019 und 2020 ist
sie nicht mehr merklich gestiegen. Dabei zeigt sich auch, dass ein immer
mehr wachsender Teil der Kontakte auf Nutzende zurückgeht, welche diese
mehrfach nutzen. Es gibt also eine wachsende Zahl von Nutzer*innen, die
immer wieder auf das Angebot zurückgreift, aber gleichzeitig immer
weniger neue Nutzer*innen. Was sich auch zeigt, ist, dass die Kontakte
sich unregelmässig über die verschiedenen Institute und Colleges der
Universität verteilen: Es gab mit allen Kontakte, aber es zeigen sich
auch die Forschungsfelder, mit denen intensive Kontakte bestehen,
nämlich der Landwirtschaftsforschung und den Geisteswissenschaften. Die
Autor*innen schliessen daraus, dass es für die Weiterentwicklung der
Services relevant wäre, sich auf die Nutzer*innen zu fokussieren, welche
die Services des FDM-Teams hauptsächlich benötigen. Andere Forschende
lösen die Herausforderungen des FDM offenbar anders, ohne die
Unterstützung der Bibliothek. Von den Angeboten am häufigsten -- mit
einer grundsätzlich über die Jahre ähnlichen prozentualen Verteilung --
wurden genutzt: Beratungsangebote, \enquote{Dataset Operation}
(\enquote{interactions in which we, as experts, do something to a
dataset} Petters et al.~2022: 12) und Visualisierungen.

Untersucht wird hier die Arbeit eines FDM-Teams in einer Universität,
aber interessant ist die Frage, ob sich Ähnliches auch für vergleichbare
Teams an anderen Hochschulen sagen lässt. Wenn ja, dann etabliert sich
FDM wohl vor allem als spezifischer Service für eine bestimmte Gruppe
von Forschungsfeldern. (ks)

\begin{center}\rule{0.5\linewidth}{0.5pt}\end{center}
\pagebreak

Stahlman, Gretchen R. (2022). \emph{From nostalgia to knowledge:
Considering the personal dimensions of data lifecycles}. In: JASIST 73
(2022) 12: 1692--1705, \url{https://doi.org/10.1002/asi.24687}

Das Argument, welches die Autorin in dieser Studie macht, ist, dass das
Teilen von Forschungsdaten eine emotionale Dimension für Forschende hat
und, dass diese Dimension in den aktuellen Modellen des
Forschungsdatenkreislaufes sowie der dazugehörigen Literatur nicht
abgebildet wird. Daraus folgt unter anderem, dass diese Dimension in der
Arbeit, die zum Beispiel von Bibliotheken in Bezug auf Forschungsdaten
geleistet wird, nicht beachtet wird, obwohl sie eine Hauptmotivation für
Forschende darstellen kann, sich überhaupt damit zu befassen, Daten
langfristig teilbar zu machen.

Die Studie basiert auf Auswertungen von Interviews mit sechs
Astronom*innen, welche im Zusammenhang der Doktorarbeit der Autorin --
die sich eigentlich mit anderen Fragen beschäftigte -- durchgeführt
wurden. In diesen zeigte sich, dass Forschende vor allem am Ende ihrer
Karrieren intrinsische Motive entwickeln, Daten langfristig nutzbar zu
machen. Diese Motive werden in der Studie zusammengefasst als Nostalgie,
Altruismus, intellektueller Anspruch, die Wissenschaft voranzutreiben
und der Wunsch, nach der Karriere ein Vermächtnis zu hinterlassen. Das
sind sichtbar andere Motive als Anforderungen von Forschungsfördern oder
der politische Wunsch nach Offenheit in der Wissenschaft, die ansonsten
in der Literatur als Motivationen für das Teilen von Forschungsdaten
angesprochen werden. Nachdem die Autorin am Anfang der Studie erwähnt
hat, dass es eine Anzahl von Modellen für den Forschungsdatenkreislauf
gibt, welche allesamt die reale Situation nicht ganz darstellen,
entwirft sie als Ergebnis ihrer Arbeit ein weiteres Modell, welches die
Motive, die sie beschrieben hat, einbezieht. (ks)

\hypertarget{bestandsmanagement}{%
\subsection{2.5 Bestandsmanagement}\label{bestandsmanagement}}

Visser, Alie (2022). \emph{Digital Bookplates: Cataloging Processes and
Workflows}. In: Cataloging \& Classification Quarterly 60 (2022) 8:
858--868, \url{https://doi.org/10.1080/01639374.2022.2148801}
{[}Paywall{]} {[}OA-Version: \url{https://ir.lib.uwo.ca/wlpub/110/}{]}

Der Artikel beschreibt, wie die Bibliotheken der Western University,
Ontario, einen Workflow etabliert haben, mit denen Personen und
Institutionen, welche den Bibliotheken Geld gespendet haben, mit
\enquote{digitalen Ex-Libris} gewürdigt werden. Diese lösten 2012
gedruckte Ex-Libris ab. Heute werden, gestaffelt nach bestimmten
Geldbeträgen und mit verschiedenen Optionen, eine bestimmte Anzahl von
Büchern digital als \enquote{gespendet von} ausgezeichnet. Interessant
ist der Artikel, weil er mit grosser Selbstverständlichkeit davon
ausgeht, dass solche Spenden und der Umgang mit ihnen etablierter Teil
des Arbeitsalltags von Bibliotheken ist. Deshalb ist die Etablierung
eines Workflows auch folgerichtig und effizient. Im DACH-Raum würde dies
wohl eher nur auf wenige Bibliotheken zutreffen. (ks)

\begin{center}\rule{0.5\linewidth}{0.5pt}\end{center}
\pagebreak

Lawal, Ibironke ; England, Mark M. (2023). \emph{One size does not fit
all: Common practices for standards collections and management}. In:
Issues in Science and Technology Librarianship (2023) 102,
\url{https://doi.org/10.29173/istl2626}

Mithilfe einer Umfrage wurde für diesen Artikel ein Spezialbereich des
Bestandsmanagements von Hochschulbibliotheken in den USA ausgeleuchtet.
Das ist recht interessant, weil es Einblick in eine alltägliche, aber
kaum in der Literatur behandelte bibliothekarische Arbeit liefert. Es
handelt sich um die Frage, ob, wie und warum diese Bibliotheken
technische Standards und Normen (also ISO-Normen und vergleichbare)
managen. Diese zeichnen sich, verglichen mit anderen Medien, durch hohe
Preise, wenige Anbieter (meist nur die Institutionen, welche die Normen
erlassen) und schlechte Zugänglichkeit aus.

Es zeigt sich, dass dies von den Bedürfnissen der Nutzer*innen und den
verfügbaren Ressourcen der Bibliotheken abhängt. Befragt wurden
Bibliotheken von Hochschulen, deren Hochschulen auch eine technische
Ausbildung anbieten, insoweit gab es schon eine Selbstauswahl. Die
Ergebnisse zeigten darauf aufbauend, dass der Einsatz von Ressourcen der
Bibliotheken für diesen Teil des Bestandes davon abhängt, wie viele
Studierende in diesen Fächern eingeschrieben sind und wie der Unterricht
organisiert ist. Je mehr Standards benötigt werden, umso eher werden sie
direkt angeschafft oder Zugänge zu ihnen lizenziert. Ansonsten werden
andere Bezugswege -- insbesondere Fernleihe und Ad-hoc Erwerb --
gewählt. Zudem gibt es einen Zusammenhang zur Grösse der Bibliothek: Je
mehr Ressourcen diese hat, umso eher baut sie vorgängig einen expliziten
Bestand von Standards auf, auch gedruckt. Je kleiner die Bibliothek,
umso eher greift sie auf Ad-hoc Besorgungen zurück. (ks)

\begin{center}\rule{0.5\linewidth}{0.5pt}\end{center}

Barr, Peter (2023): \emph{Ethical acquisitions in academic libraries: a
simple idea without a simple solution}. In: Insights: the UKSG journal
36.1:2, \url{https://doi.org/10.1629/uksg.600}

Die Universitätsbibliothek Sheffield hat 2021 eine
\enquote{Comprehensive Content Strategy} (CCS) verabschiedet, die
darlegt \enquote{how the Library provides access to the content required
for teaching and research at the University, builds, manages and shares
collections of ongoing cultural value that showcases the
University\textquotesingle s research transforms academic publishing and
scholarly communication.}
(\url{https://www.sheffield.ac.uk/library/about/content-strategy})

Nach zwei Jahren blickt Peter Barr nun -- teilweise sehr persönlich --
auf die Genese des Strategiepapiers zurück, und bespricht einige
Möglichkeiten und Grenzen ihrer Anwendung in der Praxis. Seine
Überlegungen können für das Nachdenken über die eigenen Kriterien und
Prinzipien für Bestandsaufbau und Zugänglichmachung interessant sein,
wenn man es schafft, mal aus dem alltäglichen Kleinklein aus
Erwerbungsbudgets, Lizenzierungsbedingungen, Anschaffungswünschen oder
Transformationsvertragsdetails herauszuzoomen. (vv)

\pagebreak

\hypertarget{bibliotheksmanagement}{%
\subsection{2.6 Bibliotheksmanagement}\label{bibliotheksmanagement}}

Natal, Gerald / Saltzman, Barbara (2022): \emph{Decisions, decisions,
decisions: decision fatigue in academic librarianship}. In: The Journal
of Academic Librarianship 48.1:102476.
\url{https://doi.org/10.1016/j.acalib.2021.102476} {[}Paywall{]}

\enquote{Entscheidungsmüdigkeit} ist ein Phänomen, das wohl in allen
Lebensbereichen und damit auch im Bibliothekswesen auftritt. Zwei
Kolleg:innen aus Toledo, Ohio, haben -- wohl zum ersten Mal -- eine
Umfrage dazu durchgeführt, wie verbreitet das Phänomen unter
amerikanischen Bibliothekar:innen ist, welche Aufgaben davon vor allem
betroffen sind und welche Faktoren einen Einfluss haben. Die Ergebnisse
\enquote{allow academic librarians to recognize the symptoms while
suggesting ways to overcome its effects}. Unter den Vorschlägen finden
sich unter anderem zwei Klassiker: Pause machen und mal was essen. :)

Die Literaturübersicht, der Umfrage-Fragebogen und die Analyse der
Antworten geben einen guten Einstieg ins Thema, falls jemand sich damit
einmal beschäftigen will. Ob eine Umfrage im deutschsprachigen Raum wohl
ähnliche Ergebnisse liefern würde? (vv)

\begin{center}\rule{0.5\linewidth}{0.5pt}\end{center}

Gou, Xiu ; Xu, Gordon (2023). \emph{Decision-Making in the Selection,
Procurement, and Implementation of Alma/Primo: The Customer
Perspective}. In: Information Technology and Libraries 42 (2023) 1,
\url{https://doi.org/10.6017/ital.v42i1.15599}

Diese Studie fokussiert -- wieder einmal mit einer Umfrage -- auf
Bibliotheken und Bibliotheksverbünde in den USA und Kanada, welche in
den letzten Jahren auf ein spezifisches Bibliothekssystem (das auch im
DACH-Raum verbreitete ALMA) gewechselt sind. Es geht darum, wie und von
wem die Entscheidungen dazu getroffen wurden. Im Artikel sind die
Ergebnisse umfassend ausgebreitet. Hier hervorgehoben werden soll aber
die Erkenntnis, dass es notwendig für Bibliotheken ist, gegenüber den
Anbietern solcher Software stärker und mit klaren Anforderungen
aufzutreten. Die Autor*innen betonen, dass die Bibliotheken viel mehr
Druck ausüben könnten, da die Anbieter von ihnen abhängig seien. Eine
Sammlung von Erfahrungen aus solchen Entscheidungsprozessen, wie sie mit
diesem Text vorliegt, kann dabei helfen, den Wissensvorsprung der
Anbieter (welche solche Prozesse viel öfter durchführen, als die
einzelnen Bibliotheken und Verbünde) auszugleichen. (ks)

\hypertarget{monographien-und-buchkapitel}{%
\section{3. Monographien und
Buchkapitel}\label{monographien-und-buchkapitel}}

\hypertarget{vermischte-themen-1}{%
\subsection{3.1 Vermischte Themen}\label{vermischte-themen-1}}

Leonelli, Sabina (2023) Philosophy of Open Science. {[}Preprint{]}
\url{http://philsci-archive.pitt.edu/id/eprint/21986}

Sabina Leonelli ist Professorin für Philosophy and History of Science an
der University of Exeter. Sie beschäftigt sich unter anderem mit der
Rolle von Daten in wissenschaftlichen Erkenntnisprozessen. In ihrem
neuen Buch, das vorab als Preprint erschienen ist, befasst sich Leonelli
eingehend mit Open Science. Sie vollzieht die Geschichte von Open
Science nach, betrachtet einige verbreitete Open-Science-Praktiken und
hinterfragt mögliche Auswirkungen auf wissenschaftliche
Erkenntnisprozesse. Abschließend beschreibt die Autorin eine alternative
Sichtweise auf Open Science, die dem verbreiteten Modell
gegenübergestellt wird.

Den Ursprung von Open Science sieht Leonelli als Reaktion auf die
Digitalisierung des Wissenschaftssystems und die damit einhergehende
Kommodifizierung wissenschaftlicher Erkenntnisse. Open Science stellte
anfangs den Versuch dar, finanzielle und rechtliche Hürden zu
überwinden, um Produkte, die das Ergebnis von Forschungsprozessen
darstellen, möglichst weit zu verbreiten. So sollte die Teilhabe an
wissenschaftlichen Erkenntnissen ermöglicht werden, sowie ihre
Überprüfung und Nachnutzung. Später wurde der Gegenstand sukzessive auf
weitere Problemfelder (zum Beispiel den Peer-Review-Prozess) und
Publikationstypen (zum Beispiel Lehrmaterialien) ausgeweitet.

Die Autorin stellt fest, dass inzwischen vermehrt uneingeschränkter
Zugang zu wissenschaftlichen Erkenntnissen gefordert wird, der unter
anderem sämtliche (Teil-)Produkte aller wissenschaftlichen Tätigkeiten
umfasst. Aus dieser Beobachtung leitet sie eine Gewichtung von Zielen
ab, die zuerst umfassende Transparenz vorsieht, und erst im Nachgang die
Qualitätssicherung der geteilten Produkte und schließlich Inklusion.

Anhand von Beispielen zeigt sie, wie Open-Science-Praktiken
implementiert werden können. Sie konzentriert sich dabei auf normative
Aspekte, beispielsweise die Ausgestaltung der Nutzungsvereinbarungen von
Forschungsdatenrepositorien und Vorgaben zu guter wissenschaftlicher
Praxis. In allen Beispielen zeigen sich nicht nur Chancen, sondern auch
Gefahren, beispielsweise die unbeabsichtigte Benachteiligung bestimmter
Personengruppen. (ds)

\begin{center}\rule{0.5\linewidth}{0.5pt}\end{center}

Boyadjian, Julien (2022). \emph{Jeunesses connectées: Les} digital
natives \emph{au prisme des inégalités socio-culturelles}. Villeneuve
d\textquotesingle Ascq cedex: Presses universitaires du Septentrion,
2022 {[}gedruckt{]}

Diese Studie unter Schüler*innen in Frankreich will klären, ob sich die
Vorhersagen über den Medienwandel durch «Digital Natives», welche Anfang
der 2000er-Jahre gemacht wurden, bewahrheitet haben. Dazu wurde zuerst
eine Umfrage in sozial verschieden aufgestellten Schulen durchgeführt
und dabei auch gefragt, ob die Schüler*innen es zulassen würden, wenn
die Forschenden deren Profile in Sozialen Medien (also zum Beispiel
deren Facebook- oder Twitteraccounts) auswerteten. Und zuletzt wurden
mit 35 Schüler*innen Interviews durchgeführt. Dabei achteten die
Forschenden (es waren mehrere, auch wenn das Buch nur von einem Autor
geschrieben wurde) darauf, alle in Frankreich möglichen
Bildungskarrieren abzubilden. In einem «Vorkapitel» geht der Autor lange
auf die Unterschiede zwischen den Schulen ein, beispielsweise einer
«ecole de 2\textsuperscript{e} chance», in welcher Jugendliche ihren
Schulabschluss nachholen können und einer «prepa», welche darauf
vorbereitet, einer der «grand ecoles» (die bis heute praktisch alle
besucht werden müssen, um später der politischen und wirtschaftlichen
Elite in Frankreich anzugehören).

Was sich zeigt, ist, dass es keine zusammenhängende «Generation» der
digital natives gibt, sondern dass sich die Nutzung digitaler Medien und
Social Media etabliert hat, dabei aber soziale Unterschiede in der
Mediennutzung, die es auch bei \enquote{traditionellen Medien} schon
gab, reproduziert wurden. (Dies gilt auch für die Bildungskarrieren, die
sich auf der einen Seite verbessert haben, indem sich der Zugang zu
Universitäten, Ausbildungsgängen und andere Bildungstitel seit
Jahrzehnten verbessert hat, sich auf der anderen Seite aber die soziale
Absonderung der Eliten und ihrer Kinder in den grand ecoles, erhalten
hat.) Was sich grundsätzlich verändert hat, ist, dass gedruckte
Zeitungen und Zeitschriften massiv an Bedeutung verloren haben -- aber
nicht ihre elektronischen Pendants, die weiterhin eine Hauptquelle für
die Informationsbeschaffung darstellen. Es zeigt sich auch, dass
politische Informationen oder aber konkrete politische oder
gesellschaftliche Partizipation sich durch die digitalen Medien zwar
verändert hat, aber grundsätzlich keine Veränderung dabei stattgefunden
hat, wer sich informiert oder engagiert: Es sind weiterhin «nur» hoch
politisierte Jugendliche, die zudem meisten aus sozial eher hoch
gestellten Schichten stammen. Dies führt auch zu einem bemerkenswerten
Ergebnis: Entgegen der Befürchtung, dass Jugendliche mehr mit «Fake
News» oder polarisierenden politischen Bewegungen in Berührung kommen
würden als die Gesamtgesellschaft, zeigt sich, dass nur der Teil der
Jugend, welcher sich an sich für politische Themen interessiert,
überhaupt mit diesen in Kontakt kommt. Der Rest ignoriert sie.

Was sich aus der Studie grundsätzlich ziehen lässt -- und für
Bibliotheken interessant sein sollte -- ist, dass sich in den letzten
Jahrzehnten zwar die Medienformen gewandelt, aber, dass sich die
sozialen Unterschiede bei der Mediennutzung dabei nicht gross verschoben
haben. Dies steht gegen Erwartungen, welche in den frühen 2000er-Jahren
geäussert wurden (beispielsweise zur Demokratisierung der Medien durch
das Internet), aber auch gegen Wahrnehmungen zum Medienverhalten von
Jugendlichen an sich. Sicherlich: Die Empirie in dieser Studie wurde in
Frankreich gewonnen und basiert auch auf französischen Schulstrukturen,
die es so in anderen Ländern Europas nicht gibt. Aber sie zeigt, dass es
sinnvoll ist -- auch in anderen Ländern -- empirisch nach der konkreten
Mediennutzung zu fragen und nicht einfach auf Behauptungen dazu, wie
«die Jugend es anders macht», zu vertrauen. (ks)

\begin{center}\rule{0.5\linewidth}{0.5pt}\end{center}

Beudon, Nicolas (2022). \emph{Le merchandising en bibliothèque : 50
fiches thématiques pour rendre votre bibliothèque plus inspirante}. (Le
design des bibliothèques publiques; 1) Bois-Guillaume: Klog éditions
{[}gedruckt{]}

Dieses Buch gibt eine Übersicht zu Möglichkeiten der Werbung für
Bibliotheken, aber grösstenteils für unterschiedliche Formen der
Bestandspräsentation (beschrieben mit dem im französischen
Bibliothekswesen oft genutzten Wort \enquote{valorisation}, was sowohl
als \enquote{Inwertsetzung} als auch \enquote{Aufwertung} übersetzt
werden kann). Dabei geht es um sehr konkrete Fragen, beispielsweise wie
Medien ins Regal zu stellen sind oder Informationstafeln platziert
werden können. Der Autor bespricht die verschiedenen Möglichkeiten und
ordnet sie ein wenig in den Kontext moderner Bibliotheken ein, aber in
sehr knappen Worten, die sich wohl gut als Behauptungen oder Thesen
beschreiben lassen. Seinen Wert hat das Buch vor allem dadurch, dass es
diese Möglichkeiten der Bestandspräsentation und Werbung an einem Ort
vereinigt und auch ausführlich bebildert. Auffällig ist allerdings, dass
die Auswahl nicht sehr breit aufgestellt ist: Der Autor erwähnt, dass er
übergreifende Trends aufzeigen will, führt dann aber neben Beispielen
aus Frankreich fast nur solche aus skandinavischen oder
US-amerikanischen Bibliotheken an. (ks)

\begin{center}\rule{0.5\linewidth}{0.5pt}\end{center}

Hahn, Daniela ; Hehn, Jennifer ; Hopp, Christian ; Pruschak, Gernot
(2023). \emph{Mapping the Swiss Landscape of Diamond Open Access
Journals. The PLATO Study on Scholar-Led Publishing. Report.}
\url{https://doi.org/10.5281/zenodo.7461728}

In der Studie, über die hier berichtet wird, wurde versucht, einen
Überblick über die Diamond Open Access Zeitschriften, welche in der
Schweiz publiziert oder mit schweizerischer Beteiligung herausgegeben
werden, zu zeichnen. Dabei wurden Interviews mit Beteiligten und eine
Analyse von Zeitschriften miteinander verbunden. Der nationale Fokus ist
durch die Finanzierung (swissuniversities, die Rektorenkonferenz der
schweizerischen Hochschulen, und die beteiligten
Universitätsbibliotheken) sowie die Projektbeteiligten (alle an
schweizerischen Hochschulbibliotheken angestellt) bedingt. Dieser Fokus
stösst immer an Grenzen in der über solche Grenzen hinweg vernetzten
Wissenschaftslandschaft. (Beispielsweise, um diese Art von Ego-Search zu
betreiben, ist \emph{LIBREAS. Library Ideas} nicht im Datensatz
enthalten, obgleich es zwei schweizerische Redaktionsmitglieder gibt,
dafür aber die Informationspraxis mit einem schweizerischen
Redaktionsmitglied, aber auch dem Vereinssitz in Luzern.) Auch ist
anzumerken, dass die Interviews mit Redakteur*innen solcher
Zeitschriften geführt wurden, inklusive Fragen danach, wie diese
Zeitschriften von anderen gesehen werden. Im Bericht wurden diese
Aussagen dann als Hinweis darauf verwendet, wie die Sicht der weiteren
Wissenschaftscommunity wäre. Das scheint eine weit überdehnte
Interpretation zu sein.

Trotzdem bietet der Bericht einen Einblick in die Gemeinsamkeiten dieser
Zeitschriften. Es gibt eine recht grosse inhaltliche Breite und auch
eine grosse Diversität der herausgebenden Körperschaften. Zwar
publizieren die Zeitschriften überwiegend in Englisch, aber es gibt auch
eine Breite von anders- und mehrsprachigen Publikationen. Was sich aber
immer zeigt, ist, dass die Arbeit an diesen Zeitschriften grösstenteils
unbezahlt und ehrenamtlich (oder, in der schweizerischen Terminologie,
im Milizsystem) geleistet wird. Das bezieht sich auch auf Aufgaben, die
über den reinen Redaktionsprozess hinausgehen, beispielsweise die Pflege
von Metadaten zu den Zeitschriften in Datenbanken wie dem DOAJ. Es
versteht sich deshalb, dass hinter dieser Arbeit jeweils persönliche
Überzeugungen der Redakteur*innen stehen. Das Hauptproblem für fast alle
Zeitschriften ist die nachhaltige Finanzierung. Diese ist in den meisten
Fällen nicht gesichert. (ks)

\begin{center}\rule{0.5\linewidth}{0.5pt}\end{center}

Thiele, Katja (2022). \emph{Öffentliche Bibliotheken zwischen
Digitalisierung und Austerität: Kommunale Strategien und ihre
Implikationen für die Bildungsgerechtigkeit.} (Sozial- und
Kulturgeographie, 55) Bielefeld: transcript Verlag, 2022 {[}gedruckt{]}

Dieses Buch ist, zumindest was die Seite der Bibliotheksforschung
angeht, enttäuschend. Es ist eine Dissertation in der Humangeographie,
insoweit lag der Fokus auch nicht darauf, explizit neues Wissen über
Bibliotheken zu generieren, sondern darauf, zu einem
Erkenntnisfortschritt in der Geographie beizutragen. Aber bezogen auf
Öffentliche Bibliotheken ist hier nur zu lernen, dass die Aufgaben von
Bibliotheken -- sowohl die, die sich diese selbst geben als auch die,
welche ihnen von den Kommunen gegeben werden -- vom lokalen und
nationalen Rahmen abhängen. Dies gilt auch für die konkrete Nutzung der
Bibliotheken durch die Bevölkerung.

Um zu diesem Ergebnis zu gelangen, untersucht die Autorin drei
Bibliotheken in ebenso vielen Ländern (Bonn, Leicester und Malmö), und
zwar mithilfe von Beobachtungen vor Ort, Interviews sowie einer
Einbettung in Daten zu den drei unterschiedlichen nationalen
Wohlfahrtsregimen. Allerdings scheint ihr in weiten Teilen die kritische
Distanz zum Untersuchungsgegenstand zu fehlen. Sie gibt, bezogen auf die
Bibliotheken, eher zusammenfassend wieder, was sie in ausgesuchter
bibliothekarischer Literatur findet beziehungsweise was ihr von den
Interviewpartner*innen berichtet wird. Grundsätzlich spiegelt sie den
Bibliotheken also nur wider, was diese über ihre Entwicklung in den
letzten Jahren ohnehin schon glauben. Hinzu kommt, dass es teilweise
erstaunliche Fehler in der Interpretation von Literatur oder
theoretischen Konzepten zu geben scheint. (Beispielsweise spricht die
Autorin davon, dass Öffentliche Bibliotheken sammeln würden. An einer
anderen Stellen behauptet sie, dass sich das soziologische Theorem
\enquote{Dritter Ort} und das literaturwissenschaftliche Theorem
\enquote{Dritter Raum} praktisch ergänzen würden, ohne dies zu
begründen. Es scheint ein wenig, als würde sie dies nur anführen, weil
beide Theoreme ähnlich benannt sind.) Wirklich irritierend ist aber,
dass die im Titel des Buches angesprochenen Themen wie
Bildungsgerechtigkeit oder Digitalisierung praktisch keine Relevanz für
die eigentliche Arbeit haben. Sie werden sehr knapp angesprochen, wobei
es scheint, als würde Bildung einfach mit \enquote{Zugang zu Medien und
zur Bibliothek} gleichgesetzt. In der restlichen Arbeit aber werden die
ganzen eingeführten Theoreme nicht wirklich weitergenutzt. Es ist nicht
ganz klar, warum sie überhaupt erwähnt wurden.

Kurzum: Zumindest über Öffentliche Bibliotheken, deren Aufgaben und
Nutzen, liefert dieses Buch kaum neues Wissen. (ks)

\begin{center}\rule{0.5\linewidth}{0.5pt}\end{center}

Sarah McNicol (2023). \emph{Supporting People to Live Well with
Dementia: A Guide for Library Services}. London: facet publishing, 2023
{[}gedruckt{]}

Die Aufgaben, denen sich Öffentliche Bibliotheken anzunehmen gedenken,
vermitteln oft den Eindruck, einfach immer mehr zu werden. (Es ist nicht
selten, dass das Personal dies mehr oder minder selbst zynisch bemerkt.)
Gleichwohl wird hier ein weiteres einführendes Buch angezeigt, welches
tendenziell eine weitere Aufgabe einführt, nämlich die, Menschen mit
Demenz in den Fokus bibliothekarischer Arbeit zu stellen. Die Autorin
argumentiert dabei, dass die Zahl der Betroffenen -- sowohl
(potenzielle) Nutzer*innen sowie Personal als auch Personen, die als
Familienangehörige und Freund*innen von Menschen, die an Demenz
erkranken, eine individuelle Betroffenheit haben -- wächst. (Allerdings
werden ähnliche Argumente auch für andere, \enquote{neue Aufgaben} für
Bibliotheken in ähnlichen Büchern immer wieder angeführt.)

Dennoch ist dieses Buch keine verlorene Lesezeit. Die Autorin geht all
die Themenbereiche durch, die man in ihm wohl erwarten würde: Was Demenz
ist, wie es sich auswirkt, wie Bibliotheken indirekt (zum Beispiel durch
den Aufbau des Bibliotheksraumes) oder direkt Angebote für von Demenz
betroffenen Personen machen können. Zudem gibt sie an, mit welchen
anderen Organisationen Bibliotheken zusammenarbeiten können. Alles immer
vor dem Hintergrund, dass dies ein Buch aus Grossbritannien ist, also
zum Beispiel mit Bibliotheken, für die es normal ist, für ihre
Nutzer*innen Veranstaltungen zu organisieren, die nichts mit Lesen oder
Büchern zu tun haben, oder aber mit Verweisen auf britische Gesetze.
Zudem integriert die Autorin immer wieder Erfahrungen mit ihrem Vater,
der an Demenz erkrankte und für den sie in seinen letzten Lebensjahren
sorgte.

Zu lernen ist in dem Buch aber vor allem eines, nämlich, dass das Leben
mit Demenz für alle (direkt und indirekt Betroffene, aber auch die
Gesellschaft an sich) auch gut gelebt werden kann, wenn man die
Interessen und "humanhood" der Personen, die von Demenz betroffen sind,
in den Fokus stellt. Das erfordert dann von allen Beteiligten,
aufmerksamer und verständnisvoller zu sein. (ks)

\hypertarget{bibliotheksgeschichte}{%
\subsection{3.2 Bibliotheksgeschichte}\label{bibliotheksgeschichte}}

Erünsal, İsmail E. (2022). \emph{A History of Ottoman Libraries.}
(Ottoman and Turkish Studies) Brookline: Academic Studies Press, 2022
{[}gedruckt{]}

Dieses Buch ist von grosser Relevanz. Fast die gesamte
Bibliotheksgeschichte, die im DACH-Raum wahrgenommen wird (also zumeist
die, die in Deutsch oder Englisch publiziert wurde), bezieht sich auf
Bibliotheken aus einer kleinen Anzahl von Ländern und Gesellschaften.
Ein grosser Teil der Welt wird in dieser Geschichte überhaupt nicht
repräsentiert. Das führt dazu, dass bestimmte Entwicklungen als
allgemein verbreitet angesehen werden, die historisch nur auf einen Teil
der Welt zutreffen. (Ein Beispiel wäre die Bedeutung von Bibliotheken
europäischer Klöster bei der Erhaltung von Wissen aus der europäischen
Antike.) Das Buch von Erünsal beschäftigt sich nun mit Bibliotheken in
einer anderen Gesellschaft, wenn auch teilweise geographisch in der
gleichen Region verortet wie die Bibliotheken der europäischen Antike.
Dabei stellt er dar, wie sich Bibliotheken in dieser Gesellschaft, dem
osmanischen Reich zwischen 1299 und 1922 (in europäischer Zeitrechnung)
entwickelten. Der Autor, der bis zu seiner Verrentung in der Türkei
Bibliothekswissenschaft unterrichtete, betont, dass eine solche
Geschichte bislang nicht vorliegen würde und betont vollkommen zu Recht,
dass sie deshalb notwendig wäre. Dadurch, dass das Buch auf Englisch
vorliegt, lässt es sich aber auch als Korrektiv der sonst im DACH- und
englischsprachigen Raum verbreiteten Bibliotheksgeschichte lesen. Es
zeigt, dass die Einrichtung Bibliothek nicht automatisch den Aufgaben,
Etappen und Entwicklungen unterliegt, denen sie in den von europäischen
Imperien und deren Nachfolgestaaten geprägten Gesellschaften unterlag.
Vielmehr ist ihre Entwicklung immer mit der Gesellschaft verbunden, in
denen sie etabliert werden. Grundsätzlich wünscht man sich nun ähnliche
Übersichten zur Bibliotheksgeschichte in anderen Gesellschaften.

Was die Bibliotheken, die Erünsal beschreibt, auszeichnet, ist, dass sie
allesamt als religiöse Stiftungen nach islamischem Recht errichtet
wurden. Diese Stiftungen gehörten von Beginn an zur osmanischen
Gesellschaft. Sie werden mit dem Ziel der allgemeinen Wohlfahrt
errichtet, ausgestattet mit Stiftungsurkunden und einem Besitz, der
Einkommen generiert (also beispielsweise über Mieten und Pacht), um die
Stiftungen langfristig zu unterhalten. Sie sind gedacht als Stiftungen
an Gott und können deshalb (eigentlich) nicht mehr geändert werden. Die
osmanische Gesellschaft fand in Zeiten der Modernisierung im 19.
Jahrhundert Wege, diese dennoch den wechselnden Zeiten anzupassen. Nach
der Ausrufung der Türkischen Republik nach dem Ersten Weltkrieg und
damit einhergehenden Veränderungen, beispielsweise der Einführung der
modernen türkischen Schrift, verloren diese Bibliotheken ihre Bedeutung.
Heute liegen die meisten ihrer Bücher zentral in einer Spezialbibliothek
in Istanbul. Aber bis zu diesem Zeitpunkt existierten sie teilweise über
Jahrhunderte.

Erünsal betont, dass die Informationen über die Bibliotheken sehr selten
erhalten sind. Er hat sie über Jahrzehnte zusammengetragen und stützt
sich vor allem auf die Stiftungsurkunden. Mit diesen kann er aber gut
Entwicklungen aufzeigen. Beispielsweise wurden Bibliotheken zuerst als
Teil von Moscheen und Universitäten gegründet und dann, ab dem 17.
Jahrhundert vermehrt, als eigenständige Einrichtungen. In den ersten
Jahrhunderten entliehen sie recht liberal ihre Bücher, aber ebenso ab
dem 17. Jahrhundert wurden sie mehr und mehr zu Einrichtungen, in denen
die Bücher vor Ort gelesen werden mussten. Dafür etablierten sie dann
aber auch Leseräume. Fast alle diese Bücher waren Manuskripte. Erst im
19. Jahrhundert werden auch verstärkt gedruckte Bücher aufgenommen. Ein
Grossteil der Bestände wird auch über Jahrhunderte weder ergänzt noch
ausgesondert, sondern immer nur ausgebessert. Im 18. Jahrhundert beginnt
der Staat direkt Verantwortung für die zahllosen religiösen Stiftungen
zu übernehmen, indem ein eigenes Ministerium eingerichtet wird, das sich
dann unter anderem auch für die Bibliotheken zuständig fühlt und
beispielsweise Inspektionen durchführt.

Das Buch ist in zwei Teile untergliedert. Im ersten berichtet der Autor
chronologisch von der Entwicklung der Bibliotheken, im zweiten geht er
dann noch einmal die Geschichte ausgewählter Aspekte wie den Etat, die
Services, das Personal oder den Bibliotheksraum durch. An Stellen ist
das etwas langatmig, da ganze Absätze lang immer wieder neue
Stiftungsurkunden angeführt werden. Aber ansonsten ist dies, wie gesagt,
ein wichtiges Buch, welches als Korrektiv für die bisherige
Bibliotheksgeschichtsschreibung unabdingbar ist. (ks)

\begin{center}\rule{0.5\linewidth}{0.5pt}\end{center}

Tygör, Lutz (2022). \emph{Die Potsdamer städtische Volksbücherei: Von
der Eröffnung 1899 bis zur Zerstörung der Stadtbücherei 1945.} Leipzig:
Engelsdorfer Verlag, 2022 {[}gedruckt{]}

Mit dieser Arbeit wird eine auch schon in dieser Kolumne besprochene
Arbeit zur Geschichte der Öffentlichen Bibliothek in Potsdam
fortgesetzt. (Vergleiche: Tygör, Lutz; Friebe, Reiner (2019).
\emph{Potsdamer städtische Volksbücherei: Vorgeschichte und Gründung}.)
Ging es im ersten Band darum, wann die Stadt Potsdam welche
Vereinsbibliotheken in die öffentliche Hand übernahm, geht es in diesem
Band nun darum, die Entwicklung von dieser Übernahme 1899 bis zum Ende
des Nationalsozialismus zu schildern. Dabei greift der Autor
hauptsächlich auf die in Potsdamer Archiven greifbaren Akten sowie
zeitgenössische Zeitschriften- und Zeitungsartikel zurück. Gleichzeitig
fliessen Funde aus den Magazinen der heutigen Stadtbibliothek (vor allem
Besitzstempel) und Interviews mit Personen, welche die Bibliotheken noch
benutzt hatten, mit ein. Präsentiert wird das Ganze chronologisch. Es
ist eine lokalhistorische Arbeit, mit all ihren Vorzügen und Nachteilen.
Der Autor nennt zum Beispiel aus den Unterlagen ständig Geldwerte,
beispielsweise zum Etat der Bibliothek, aber ohne dass diese irgendwie
eingeordnet und damit in ihrer Höhe verständlich würden. Er kann aber
auch Angaben zur Biographie verschiedener Bibliothekar*innen machen.

Auffällig ist an dem Buch, dass bis heute grundsätzlich eine
übergreifende Geschichte des Öffentlichen Bibliothekswesens im DACH-Raum
fehlt. Dort, wo es solche Literatur gibt -- also vor allem zur
Geschichte der Bibliotheken während des Nationalsozialismus -- greift
der Autor auf diese zurück und ordnet die Entwicklung in Potsdam in
diese ein. Aber für die Zeit vorher scheint er teilweise nicht den
notwendigen Überblick über die allgemeinen Entwicklungen der
Volksbibliotheken zu haben und interpretiert die Potsdamer Situation
nicht als Teil dieser Entwicklung. Beispielsweise ist er an
verschiedenen Stellen erstaunt, dass die Volksbücherei die Ausleihe auf
ein bis zwei Bücher pro Person beschränkte, obgleich dies mindestens bis
in die 1950er-Jahre hinein der Normalfall war (da Volksbibliotheken die
Leser*innen nicht nur gegen das Lesen \enquote{falscher Literatur}
erziehen wollten, sondern auch zum \enquote{richtigen Lesen}, zu dem
auch das langsame und genaue Lesen gehörte). Zudem betont er, dass eine
zweite Bibliothek, die er bespricht -- die der Gemeinde Nowawes, die
dann in Babelsberg umbenannt und anschliessend in Potsdam eingemeindet
wurde -- im sogenannten Richtungsstreit der 1910er- bis 1920er-Jahre der
\enquote{Leipziger Richtung} gefolgt wäre, nennt dann aber als
Auswirkung dieser Orientierung nach Leipzig solche Dinge wie den Druck
von \enquote{Leserkatalogen}, welche für alle Volksbüchereien der
damaligen Zeit normal waren, egal zu welcher Richtung oder Bewegung sie
sich zählten.

Grundsätzlich zeigt der Autor -- wie es auch schon im vorhergehenden
Buch sichtbar wurde --, dass sich die Bibliothek in Potsdam (und auch
die in Nowawes) so entwickelten wie viele Bibliotheken in der damaligen
Zeit. Sie war weder sonderlich innovativ noch irgendwie im Hinterfeld,
auch wenn die Bibliothek in Nowawes etwas besser organisiert war als die
in Potsdam. Als interessant hervorzuheben ist, dass in Potsdam die
Volksbücherei und die Lesehalle als zwei voneinander getrennte
Einrichtungen geführt wurden, die auch in unterschiedlichen Gebäuden
untergebracht waren. (ks)

\begin{center}\rule{0.5\linewidth}{0.5pt}\end{center}

Bassett, Troy J. (2020). \emph{The Rise and Fall of the Victorian
Three-Volume Novel.} (New Directions in Book History) Cham: Palgrave
Macmillan, 2020 {[}gedruckt{]}

Romane, die in drei Bänden veröffentlicht wurden, waren in der
britischen Publikationslandschaft des 19. Jahrhunderts prägend und sind
es auch in der Forschung zu diesem, viktorianischen Zeitalter. Und --
deshalb wird das Buch hier angezeigt -- sie waren eng mit den
\enquote{lending libraries} dieser Zeit verbunden. Diese Bibliotheken
waren Unternehmen, welche das Verleihen von Büchern als Geschäft
betrieben. In Grossbritannien gab es damals eine Reihe von breit
aufgestellten Firmen, welche Netze dieser Libraries unterhielten. Sie
stellten für die wachsende Schicht von Leser*innen aus dem Mittelstand
eine wichtige Möglichkeit dar, um Zugang zu Literatur zu erlangen, und
werden zum Beispiel auch in zahlreichen Romanen dieser Zeit erwähnt.
Erst Anfang des 20. Jahrhunderts endete die Zeit dieser Bibliotheken.
Einerseits wurden sie immer schlechter beleumundet (in ähnlicher Weise,
wie dies im deutschen Sprachraum mit dem \enquote{Kampf gegen Schmutz
und Schund} geschah), andererseits wurde ihr Geschäftsmodell immer
weniger profitabel. Aber dadurch, dass es eine starke
Forschungstradition zur viktorianischen Literatur gibt, ist über diese
lending libraries auch weit mehr bekannt als über die zeitgleich im
deutschen Sprachraum betriebenen \enquote{gewerblichen
Leihbibliotheken}.

Der dreibändige Roman galt als prototypisch für die von den lending
libraries angebotene Literatur. Einerseits wird in der Forschung oft
davon gesprochen, dass die lending libraries ihr Geschäftsmodell auf
diesen aufbauten (drei einzelne Bücher auszuleihen war teurer als drei
auf einmal) und deshalb kontinuierlich neue dreibändige Romane benötigt
hätten. Andererseits gab es 1894 einen offenen Brief mehrerer dieser
Bibliotheksfirmen an die britischen Verlage, in welchem sie ein Ende
dieses Genres forderten und ankündigten, es ab 1895 nicht mehr zu
kaufen. Anfang des 20. Jahrhunderts erschienen solche Werke dann
tatsächlich praktisch nicht mehr, was in der Forschung auf diesen
\enquote{Boykott} der Bibliotheken zurückgeführt wird.

Während andere Forschungen zum dreibändigen viktorianischen Roman sich
nun auf die inhaltlichen Aspekte der Romane konzentrieren, sich auf das
mit der Zeit wandelnde Bild dieser Publikationen in der britischen
Gesellschaft fokussieren oder diese Romane als Teil der sich wandelnden
Publikations- und Leselandschaft untersuchen, ist diese Arbeit hier an
den ökonomischen Aspekten interessiert. Es geht um die Anzahl der
publizierten dreibändigen Romane, um Fragen der Kosten, Zahlungen und
Profite sowie um die ökonomische Entwicklung von Verlagen und lending
libraries. Der Autor erstellte einen Datensatz aller nachweisbaren
dreibändigen Romane, die im 19. Jahrhundert in Grossbritannien
erschienen sind, inklusive der verfügbaren Angaben zu Preisen oder
Auflagenhöhe. Ausserdem nutzte er zur Ergänzung seiner Daten die
Geschäftsarchive eines grossen Verlages und einer grossen
Bibliotheksfirma. Was er damit zeigen kann, ist, dass tatsächlich die
Bibliotheken Hauptabnehmer der Romane waren (die zumeist in Auflagen von
500 bis 1000 Stück erschienen und dann en gros an die Bibliotheksfirmen
verkauft wurden), dass die Bibliotheken grundsätzlich profitabel waren
und dabei auch nicht von anderen Entwicklungen im Publikationsmarkt
(insbesondere billigen Nachdrucken, seriell in Literaturzeitschriften
publizierten Romanen oder einbändigen Werken, die allesamt während des
19. Jahrhunderts in immer grösserer Zahl erschienen) beeinflusst waren
sowie dass das \enquote{Ende} der dreibändigen Romane eher aus
allgemeinen Entwicklungen auf dem Literaturmarkt zu erklären ist als mit
dem Ultimatum der Bibliotheken von 1894. Dieses war eher Ausdruck des
Wandels als Grund dafür. Zudem zeigt er, dass der dreibändige Roman für
weibliche Autorinnen eine Möglichkeit innerhalb der viktorianischen
Gesellschaft darstellte, sich zu profilieren. Sie wurden grundsätzlich
-- wenn auch nicht von allen Verlagen -- eher verlegt als männlichen
Autoren, und ihnen wurden für ihre Manuskripte im Durchschnitt auch
höhere Summen gezahlt. (ks)

\begin{center}\rule{0.5\linewidth}{0.5pt}\end{center}

Weigand, Jörg (2018). \emph{Träume auf dickem Papier: Das Leihbuch nach
1945} -- \emph{ein Stück Buchgeschichte}. (2., erweiterte Auflage)
Baden-Baden: Nomos Verlagsgesellschaft, 2018 {[}gedruckt{]}

Zu dem hier direkt zuvor besprochenen Buch passt die Publikation von
Jörg Weigand. Er beschäftigt sich mit den Büchern, welche in
(gewerblichen) Leihbüchereien in der Bundesrepublik Deutschland entlehnt
wurden. Wie in der Vorstellung des letzten Buches erwähnt, ist die
Geschichte dieser Leihbüchereien im DACH-Raum bislang kaum erforscht.
Mit diesem Buch liegt immerhin ein Anfang dazu vor. Nach 1945 gab es bis
in die frühen 1970er-Jahre einen Aufschwung solcher Leihbüchereien,
allerdings wohl nur in der Bundesrepublik Deutschland. Dazu gehörten
nicht nur die Büchereien selbst, sondern auch ein ganzes System aus
Verlagen, Druckereien, Autor*innen und dazugehöriger Dienstleister*innen
(Grafik, Lektorat, Satz, Vertrieb und so weiter). Diese Einschätzung ist
ohne weitere Forschung mit Vorsicht zu geniessen. Weigand gibt aber an,
dass er bei seiner Recherche rund 220 deutschsprachige
\enquote{Leihbuch-Verlage} nachweisen konnte, davon aber nur ganz wenige
in der Schweiz und keine in Österreich oder anderen Ländern.

Der Fokus des Buches sind die eigentlichen Leihbücher. Diese wurden
explizit für die Leihbüchereien geschrieben und produziert. Es gab einen
ganzen Kosmos von Autor*innen und Verlagen, die nur in diesem System
engagiert waren. Daneben gab es ein ähnliches System für Romanhefte, die
im Bahnhofsbuchhandel vertrieben wurden und weiterhin das System,
welches für den Buchhandel (und die Öffentlichen Bibliotheken)
produzierte. Es gab kaum Überschneidungen zwischen diesen Systemen. Nur
einige Werke wurden in mehreren Formen (Leihbuch, Romanheft,
\enquote{richtiges Buch}) publiziert; nur sehr wenige der Autor*innen
gelang es, den Übergang von einem zum anderen System zu meistern. Es gab
auch andere Herausforderungen zu meistern: Der Leihbuchmarkt verlangte
ständig neue, aber gewissermassen auch \enquote{eingespielte} Romane. Es
ging deshalb auch um Quantität. Autor*innen schrieben in ihrer Hochphase
ein bis zwei Romane pro Monat, allerdings praktisch nie unter ihrem
eigenen Namen, sondern unter zahlreichen Pseudonymen, wobei viele
\enquote{Verlagspseudonyme} waren, die vom Verlag mehreren Autor*innen
zugewiesen wurden, um Reihen zu veröffentlichen.

Im Buch schildert der Autor dieses System, inklusive der Autor*innen und
Verlage, der Produktionsbedingungen, der Genres der Leihbücher, aber
auch der Indexierung vieler dieser Bücher unter dem Label des
Jugendschutzes, welcher ab Mitte der 1950er-Jahre gesetzlich und
institutionell geregelt wurde. Für die Bibliotheksgeschichte ist
anzumerken, dass die eigentlichen Leihbüchereien oder gar die
Leser*innen praktisch nicht behandelt werden. Den Büchereien sind nur
sieben Seiten gewidmet, auf denen sogar einige Zahlen geliefert werden
(der Autor spricht von über 27.000 dieser Büchereien, die 1960 existiert
haben sollen, teilweise organisiert in Firmen mit zahlreichen Filialen).
Aber: Das gesamte Buch ist journalistisch angelegt, nicht als
wissenschaftliche Publikation. Es gibt ein Quellenverzeichnis, aber
keine direkten Literaturnachweise. Deshalb ist nicht ersichtlich, wo
diese Zahlen herstammen.

Grundsätzlich ist anzumerken: Der Autor ist Schriftsteller, Herausgeber
und versteht sich wohl als -- positiv konnotiert --
\enquote{Laienforscher}. Er publiziert kontinuierlich, unter anderem zum
Thema Leihbücher, und erinnert in gewisser Weise selbst an die
Autor*innen der Leihbücher, die er beschreibt. Zumindest lesen sich
Teile des Buches so: Routiniert niedergeschrieben, eingängig formuliert,
aber gleichzeitig auch so, dass Nachweise fehlen, Fakten und subjektive
Einschätzungen (gerade dann, wenn einzelne Romane vorgestellt werden)
unentwirrbar miteinander verwoben sind und viele Geschichten wiederholt
werden, teilweise in den gleichen Worten. Es liest sich in weiten Teilen
wie eine Artikelsammlung eines Fans. Interessant ist das Buch trotz
dieser Einschränkungen, zumal es auf vielen Interviews mit Beteiligten
des \enquote{System{[}s{]} Leihbuch} und einem grossen Spezialwissen des
Autors über diese Bücher aufbaut. Es liefert Einblick in einen Bereich
von Bibliotheken (wenn man sie als Institutionen versteht, die Bücher
zur Ausleihe anboten), welcher in der deutschsprachigen
Bibliotheksgeschichte (und, soweit zu sehen, auch der zeitgenössischen
bibliothekarischen Literatur) bei Weitem nicht die Beachtung erhalten
hat, die er verdient. Schon, weil er zu seiner Hochzeit wohl für
hunderttausende Menschen einen der Hauptzugänge zu Literatur darstellte.
Hier scheint sich die Haltung, dass Leihbücher keine \enquote{richtige
Literatur} darstellten, reproduziert zu haben. Es ist zu hoffen, dass
sich in Zukunft nicht nur engagierte Fans wie der Autor mit diesem
Bereich beschäftigen. (ks)

\begin{center}\rule{0.5\linewidth}{0.5pt}\end{center}

Gallo, Daniela ; Provost, Samuel (dir.) (2018). \emph{Nancy-Paris:
1871-1939. Des bibliothèques au service de l'enseignement universitaire
de l'histoire de l'art \& de l'archéologie}. {[}Paris{]}: Éditions des
Cendres, 2018 {[}gedruckt{]}

Diese Aufsatzsammlung beschäftigt sich tatsächlich, wie im Titel
angekündigt, mit den Bibliotheken in Nancy Ende des 19. und Anfang des
20. Jahrhunderts. Aber das eigentliche Hauptthema ist die Karriere
einiger wichtiger Persönlichkeiten um das damals an der Universität in
Nancy aufgebaute Institut de l'archélogie et l'histoire de l'art. Diese
Universität -- heute Teil der Université de Lorraine, an welcher die
Herausgeber*innen und Autor*innen arbeiten -- war in der Position einer
«provenziellen» Universität, die im Schatten gleich mehrerer anderer
Einrichtungen stand. Zum einen im Schatten der Einrichtungen in Paris,
also dem intellektuellen Zentrum Frankreichs. Zum anderen aber seit 1871
auch im Schatten der Universität Strassburg, welche nach dem Anschluss
des Elsass an das Deutsche Reich -- im Buch konsequent Temps de Annexion
genannt -- als preussische «Vorzeigeuniversität» aufgebaut wurde. Die
Universität in Nancy -- als Hauptstadt Lothringens / Lorraines
potentiell das französische Gegenstück zu Strassburg -- arbeitete sich
in gewisser Weise an diesen beiden Schatten ab. Die Einrichtung des
genannten Instituts, welches unter anderem die Geschichte Ostfrankreichs
bearbeitete und dabei immer das Elsass mit einbezog, war Teil dieses
Versuchs, als Universität eine eigene Position aufzubauen -- aber
gleichzeitig war dessen Arbeit ohne die Verbindungen nach Paris und
Strassburg überhaupt nicht möglich.

Die Bibliotheken der Universität -- die zwar zentralisiert werden
sollten, aber dann doch immer wieder auch neu als Spezialbibliotheken
gegründet wurden -- werden im Buch mit dargestellt. In einigen Texten
geht es um die Bibliotheken, welche Professoren selbst aufbauten und
dann teilweise der Universität vermachten (oder die am Ende doch an
Universitäten in Paris oder, nach 1918, nach Strasbourg gingen). Ein
Text beschreibt die parallelen Entwicklungen von Universitätsbibliothek
und Stadtbibliothek in Nancy. Teilweise wird auch geschildert, wie
wichtig einzelne Professoren die Arbeit in ihren oder anderen
Bibliotheken fanden, insbesondere für das damals neu begründete
Fachgebiet der Kunstgeschichte. Aber das alles ist in einem eher
kursorischen Modus geschrieben: Es werden einzelne Schritte der
Bibliotheksentwicklungen genannt, es werden aus Akten auch immer wieder
Geldsummen genannt, welche für Bibliotheken aufgewandt wurden,
allerdings ohne diese jeweils einzuordnen. Es wird mit relativ vielen
Photos gearbeitet, die einen Eindruck der damaligen Arbeitsverhältnisse
in Bibliotheken vermitteln. Aber ansonsten hinterlässt das Buch eher den
Eindruck einer recht unfokussierten Regionalgeschichtsschreibung. (ks)

\begin{center}\rule{0.5\linewidth}{0.5pt}\end{center}

Ramtke, Nora (2022). \emph{Miszellen zur Geschichte der Zeit: Zu Format,
Materialität und Temporalität historisch-politischer Journale
1813-1815}. In: Gretz, Daniela ; Krause, Marcus ; Pethes, Nicolas
(Hrsg.) (2022). Miszellanes Lesen: Interferenzen zwischen medialen
Formaten, Romanstrukturen und Lektürepraktiken im 19. Jahrhundert.
(Journalliteratur ; 5). Hannover: Wehrhahn Verlag, 2022, 97--120
{[}gedruckt{]}

Der Sammelband, in welchem dieser Beitrag erschienen ist, beschäftigt
sich mit Miszellen in Zeitschriften und Zeitungen des 19. Jahrhunderts,
einem Genre, den man «Vermischtes» nennen könnte. Dabei geht es sowohl
um Rubriken, die mal mit mehr, mal mit weniger Plan Meldungen
versammelten, sowie um eigene Publikationsformen wie Almanachen für den
Massenmarkt, als auch um die Entwicklung des Lesens und der Lesetheorien
mit Bezug auf diese Miszellen. Grundsätzlich werden Miszellen dabei als
Phänomen des 19. Jahrhunderts verstanden, das einerseits in der Zeit der
sich etablierenden Massenmedien entstand und damals schon zu
verschiedenen Auseinandersetzungen, beispielsweise Fragen von
pädagogischen Gefahren oder Nutzen, führte. Und anderseits werden sie
verstanden als ein heute für die Forschung schwierig zu fassendes, weil
vielgestaltiges Thema.

Im Beitrag stellt die Autorin nun ein spezifisches Genre vor, welches
diesen Suchprozess der Massenmedien hin zu einer etablierten Form
symbolisiert: Sammlungen «zur Geschichte der Zeit», welche 1813--1815
(der Zeit der Befreiungskriege gegen Napoleon) erschienen. Diese wurden
gerade in den deutschen Staaten in grosser Zahl aufgelegt, immer als
eigenständige Publikationen, aber zumeist nur in wenigen Nummern. Sie
wollten auf der einen Seite alles sammeln, was die damalige Zeit
irgendwie abbildete: Dokumente, Essays, Berichte, Gedichte, Flugblätter.
Das hatte manchmal mehr, oft weniger klare Struktur. Sie verblieben auch
formal im Ungefähren: Sie wollten keine schnell erscheinenden Zeitungen
sein, aber auch keine schon fertige Geschichte. Vielmehr wollten sie
Beiträge versammeln, aus denen später, wenn die aufregenden Zeiten
vorbei wären, Geschichte geschrieben werden könnte. Sie erschienen auch
zumeist in Fortsetzung, also nicht als einzelne Nummer, aber
gleichzeitig in -- so hiess es -- «ungezwungener Weise». Und: Teilweise
wurden sie schnell sang- und klanglos wieder eingestellt. Aber manchmal
wurden sie auch (nach der Restauration, inklusive der Restauration der
Pressezensur, gleichzeitig auch dem endgültigen Ende von Napoleon) in
reguläre Zeitschriften überführt. Das ganze Genre ist also nicht klar zu
fassen.

Was den Beitrag hier interessant macht, ist, dass die Autorin, um diesen
Punkt zu machen, auf die Katalogisierungspraxis von Bibliotheken
zurückgreift. Sie zeigt, dass auch Bibliotheken mit ihren sonst so
klaren Abgrenzungen keine klare Einschätzung dieses Genre treffen
können: Die gleiche «Sammlung» steht in einigen Bibliotheken als
Zeitschrift, in anderen als Fortsetzungswerk und wieder in anderen --
vor allem, wenn sie irgendwann einmal in einen Band gebunden wurden --
als eigenständiges Werk, was von der Autorin als Hinweis darauf
verstanden wird, dass das Genre sich nicht wirklich fassen lässt. (ks)

\begin{center}\rule{0.5\linewidth}{0.5pt}\end{center}

Schrott, Georg (2022). \emph{Barocke Klosterbibliotheken als
\enquote{Schauräume}: Überlegungen zu einigen Implikationen dieses
Begriffs}. In: Jahrbuch für Buch- und Bibliotheksgeschichte 7 / 2022.
Heidelberg: Universitätsverlag Winter, 2022: 103--122. {[}gedruckt{]}

In seinen Überlegungen liefert Schrott grundsätzliche Fragen dazu, ob
und wie barocke Klosterbibliotheken gelesen werden können. Er betont,
dass sie explizit als Repräsentationsräume gebaut wurden. Allerdings
nicht für ein Massenpublikum, sondern für eine kleine Anzahl von
Männern. Ihr Bild- und Raumprogramm sei nicht zufällig, sondern explizit
ausgewählt. Zudem seien die Räume auch gewiss gezeigt worden, also nicht
einfach frei zu betreten gewesen. Der Blick und Schritt der Besucher sei
in ihnen gelenkt gewesen. Der Artikel fokussiert auf die genannten
Bibliotheken, aber seine Überlegungen lassen sich auch auf andere
Bibliotheken und andere Zeitepochen beziehen. (ks)

\begin{center}\rule{0.5\linewidth}{0.5pt}\end{center}

Domanski, Kristina (2023). \emph{Leselust im spätmittelalterlichen
Basel.} (201. Neujahrsblatt Gesellschaft für das Gute und Gemeinnützige
Basel). Basel: Schwabe Verlag, 2023 {[}gedruckt{]}

Die GGG (Gesellschaft für das Gute und Gemeinnützige Basel) ist eine
Stiftung, welche in Basel-Stadt eine ganze Anzahl von kulturellen und
sozialen Aufgaben übernommen hat, beispielsweise den Betrieb der
Öffentlichen Bibliotheken. Eine andere Aufgabe ist die
\enquote{Förderung der Geschichtskenntnisse über Basel}
(\url{https://www.ggg-basel.ch/ggg-organisationen/}), was unter anderem
durch die jährliche Publikation einer Monographie (dem
\enquote{Neujahrsblatt}) geschieht. Die aktuelle Ausgabe beschäftigt
sich mit einer Büchersammlung, welche hauptsächlich in der Zeit der
Wiegendrucke, also in der zweiten Hälfte des 15. Jahrhunderts, vom
Basler Ehepaar Barbara zum Luft und Niklaus Meyer zum Pfeil
zusammengetragen wurden. Diese befindet sich heute in der
Universitätsbibliothek (in ihrer Funktion als Kantonsbibliothek). Die
Sammlung besteht aus gerade einmal neun Büchern, teilweise gedruckt,
teilweise per Hand abgeschrieben und das zum Teil auch nicht
vollständig. Die abgeschriebenen Bücher sind teilweise illustriert.

Die Autorin nimmt diese Sammlung zum Anlass, grundsätzlich in die
Produktion, Verbreitung und Nutzung von Literatur in dieser Zeit des
Medienwandels einzuführen. Insbesondere geht sie auf die verschiedenen,
nebeneinander bestehenden Formen der Produktion und Verbreitung von
Büchern ein -- beispielsweise, wie sich Druckereien etablierten, wie der
Buchhandel organisiert war, aber auch, wer Illustrationen in
handgeschriebenen Manuskripten vornahm. Die neun Bücher eignen sich
dafür, weil sie einen Einblick in das Leben eines wohlhabenden, aber
nicht zu exponierten Ehepaares erlauben. In der Sammlung stehen Romane
neben Werken der religiösen Erbauung. Das Neujahrsblatt ist graphisch
aufwändig gestaltet, beispielsweise werden Teile aller Bücher ganzseitig
reproduziert. Es bewegt sich, dem Auftrag entsprechend, den sich die GGG
mit dieser Reihe gegeben hat, auf der Grenze zwischen streng
wissenschaftlicher und populärer Geschichtsschreibung. (ks)

\hypertarget{social-media}{%
\section{4. Social Media}\label{social-media}}

Mastodon versus Twitter / @libreas@openbiblio.social /
\href{https://openbiblio.social/@libreas}{https://openbiblio.social/@libreas}
/ \url{https://openbiblio.social/about}

Durch die Übernahme von Twitter durch Elon Musk und seiner Politik,
nicht nur den Dienst umzubauen, dabei, wie ein villain billionaire aus
einem Comic der 1980er-Jahre keine Rücksicht auf das Überleben seiner --
grösstenteils -- ehemaligen Angestellten zu nehmen und gleichzeitig alle
in den letzten Jahren mühsam erkämpften Moderationsregeln und
Ausschlüsse von schrecklichen Personen wieder rückgängig zu machen, kam
es im Bibliothekswesen Ende 2022 zu einem regelrechten Auszug aus diesem
Dienst. Schon lange gab es die -- vom moralischen und politischen
Standpunkt wohl immer schon überlegene -- Alternative Mastodon, für die
auch bereits eine ganze Reihe von Servern betrieben wurden.

Man weiss nicht, wie sich die gesamte Sache entwickelt hat, wenn diese
Kolumne erscheint. Wird Twitter dann abgeschaltet sein? Wird der Dienst
jemand anderem gehören? Wird sich Mastodon durchsetzen, zumindest als
Social-Media-Dienst für das Bibliothekswesen im DACH-Raum? Oder
vielleicht ein anderer Dienst? Die LIBREAS erscheint jetzt im
neunzehnten Jahr und hat seitdem verschiedene Social-Media-Plattformen
kommen sehen (und auch ausprobiert). Auf wenigen davon war sie so aktiv
wie auf Twitter (was allerdings auch nicht überschätzt werden sollte).
Aber auch diese Redaktion hat jetzt lieber einen Mastodon-Auftritt
eingerichtet (wie auch viele Mitglieder der Redaktion selbst) und zwar
auf dem Server openbiblio.social, der betrieben wird von Kolleg*innen
der Staatsbibliothek Preußischer Kulturbesitz Berlin, und zur Heimat
vieler Accounts der bibliothekarischen Fachcommunity wurde. (ks)

\hypertarget{konferenzen-konferenzberichte}{%
\section{5. Konferenzen,
Konferenzberichte}\label{konferenzen-konferenzberichte}}

{[}Diesmal keine Beiträge{]}

\hypertarget{populuxe4re-medien-zeitungen-radio-tv-etc.}{%
\section{6. Populäre Medien (Zeitungen, Radio, TV
etc.)}\label{populuxe4re-medien-zeitungen-radio-tv-etc.}}

Krymalowski, Sarah (2022). \emph{Iqaluit library closed for weeks due to
a lack of staff}. CBC North,
\url{https://www.cbc.ca/news/canada/north/iqaluit-library-closed-to-public-november-2022-1.6663470}

Zwischen Oktober 2022 und Januar 2023 war die Öffentliche Bibliothek in
Iqaluit (Nunavut, Kanada) geschlossen. Der Bericht des öffentlichen
Rundfunks CBC stellt dar, wie es dazu kam: Die Regierung von Nunavut war
nicht in der Lage, für diese Zeit eine qualifizierte Person zu finden,
um die Bibliothek zu leiten. Die vorherige war in den Ruhestand
gegangen, die folgende trat ihren Dienst erst Anfang 2023 an. Die
Nachricht hinterlässt einen bitter-süssen Eindruck. Einerseits heisst
dies, dass die Bibliothek geschlossen war -- wobei ihre Nutzer*innen
wohl nicht einfach auf andere Bibliotheken ausweichen konnten, da alle
Siedlungen in Nunavut weit voneinander entfernt liegen --, andererseits
betonte die Regierung in ihrer Antwort an CBC, dass sie darauf achtet,
dass nicht jede beliebige Person eine Bibliothek führen könne, sondern
dass dafür eine bestimmte Qualifikation notwendig sei. (ks)

\begin{center}\rule{0.5\linewidth}{0.5pt}\end{center}

Pistachio, George (2022): Inside Elise By Olsen's State-Of-The-Art
Fashion Research Library in Oslo. In: AnOther. 02.12.2022
\url{https://www.anothermag.com/fashion-beauty/14552/elise-by-olsen-interview-international-library-of-fashion-research}

Elise by Olsen, Wunderkind des Modejournalismus, eröffnete am
norwegischen Nationalmuseum in Oslo eine International Library of
Fashion Research. Das Anliegen der Bibliothek ist, eine umfassende
Sammlung an moderelevanter Literatur und Forschungsobjekten
zusammenzutragen. Zu letzteren zählen unter anderem Pressemitteilungen
der Modehäuser, die wiederum bis März 2023 in der Eröffnungsausstellung
\enquote{For Immediate Release: The Art of the Press Release} gezeigt
werden. In dieser wird die Praxis des Werbetextens in der Entwicklung
von den 1970er-Jahren bis heute und damit das Phänomen der
Modekommunikation auch unter dem Einfluss kommunikationstechnologischer
Entwicklungen untersucht. Die Bibliothek selbst wird stark
printorientiert sein. (bk)

\begin{center}\rule{0.5\linewidth}{0.5pt}\end{center}

Lee, David (2022): \emph{Has relying on tech made us stupid?} In:
Financial Times, 26./27.11.2022, Life \& Arts, S. 2 {[}gedruckt{]}

Der Autor weist -- mit dem Hinweis auf einen Artikel in Vice -- auf die
Beobachtung hin, dass Studierende GPT-3 (GPT=\emph{Generative
Pre-trained Transformer}) oder andere Varianten Künstlicher Intelligenz
nutzen, um die Grundstruktur für Studienarbeiten vorformulieren zu
lassen. Da die KI syntaktisch einmalige Texte produziert, können
aktuelle Softwarelösungen zur Plagiatserkennung dies nicht erkennen. Im
Artikel wird am Rande die Frage aufgeworfen, ob es sich beim Einsatz von
KI-Textproduktion um Betrug im traditionellen Sinn handelt. Zudem wird
der Experte Nathan Bescher, selbst Anbieter einer KI-Lösung, mit dem
Hinweis zitiert, diese automatischen Formen der Textproduktion mit
Vorsicht in Journalismus und Wissenschaft einzusetzen, da die KI derzeit
auch dazu neigt, falsche Angaben zu produzieren. (bk)

\begin{center}\rule{0.5\linewidth}{0.5pt}\end{center}

Schuessler, Jennifer (2023): \emph{For Rare Book Librarians, It's Gloves
Off. Seriously.} In: New York Times, 09.03.2023,
\url{https://www.nytimes.com/2023/03/09/arts/rare-books-white-gloves.html}

Als die New York Times in einem Artikel Mitte Februar ein Foto eines
wertvollen Manuskriptes zeigte, das mit bloßen Händen angefasst wurde,
erschienen prompt einige Kommentare, die das Fehlen von
Baumwollhandschuhen kritisierten. In dieser Replik macht sich nun
Jennifer Schuessler daran, \enquote{the general public's unshakable --
and often vehemently expressed -- belief that old books should be
handled with Mickey Mouse-style white cotton gloves}, richtigzustellen.
Dies tut sie auf sehr lesenswerte und unterhaltsame Weise. Während für
bewanderte Altbestandsbibliothekar*innen inhaltlich nicht viel Neues zu
finden sein wird, so verlinkt Schuessler doch auf recht viele
interessante Informationen zum Thema -- von wichtigen wissenschaftlichen
Aufsätzen bis zu einschlägiger Star Wars Fanfiction. (eb)

\begin{center}\rule{0.5\linewidth}{0.5pt}\end{center}

Rust, Martje (2023): \emph{Nicht nur was für Bücherwürmer}. In: KATAPULT
MV. 06.04.2023,
\url{https://katapult-mv.de/artikel/nicht-nur-was-fuer-buecherwuermer}

Im April 2023 nimmt die Stadtbibliothek Greifswald Saatgut in ihren
Bestand und wird damit Saatgutbibliothek. Sie ist eine von insgesamt
sieben Bibliotheken in Mecklenburg-Vorpommern, die ihren Nutzenden ein
entsprechendes Angebot bieten. In Greifswald geht es ums Gemüse
beziehungsweise genauer um die Sorten Tomaten, Salat, Bohnen, Erbsen und
Rote Gartenmelde. Sofern die Saatgutnutzung erfolgreich ist, werden die
Nutzenden gebeten, Samen nach der Ernte an die Bibliothek zurückzugeben.
(bk)

\begin{center}\rule{0.5\linewidth}{0.5pt}\end{center}

Hong, Jackie ; MacIntyre, Chris (2023). \emph{Concerns raised after
\textquotesingle blatantly transphobic\textquotesingle{} book labelled
as staff pick at Whitehorse library}. In: CBC North, 29. April 2023,
\url{https://www.cbc.ca/news/canada/north/irreversible-damage-whitehorse-library-pick-1.6826815}

In den aktuellen Kampagnen gegen bestimmte Bücher, Themen und
Veranstaltungen in Bibliotheken, die von rechtsgerichteten Kreisen in
den USA und Kanada geführt werden, geht es zumeist darum, diese
grundsätzlich aus den Bibliotheken zu entfernen. Bibliotheken in diesen
Ländern, die sich als Orte der freien Information verstehen, halten oft
dagegen, nicht selten mit direkter Unterstützung aus ihrer Community.

Ein Artikel aus Whitehorse, Yukon, zeigt exemplarisch auf, dass andere
gesellschaftliche Richtungen eben nicht das reine Gegenbild zu diesen
Kampagnen und Bewegungen darstellen und vielleicht einfach alle Medien
gleich behandelt sehen wollen, sondern weit differenzierter
argumentieren. In der Öffentlichen Bibliothek in Whitehorse (Yukon,
Kanada) wurde ein Buch nicht nur in den Bestand eingestellt, sondern
explizit als \enquote{Staff pick} hervorgehoben, welches als explizit
transphob bezeichnet wird. Eine Anzahl von Personen und Organisationen
äusserten sich daraufhin besorgt, auch die Bibliothek reagierte und will
nun klären, warum ein solches Buch gesondert hervorgehoben wurde. Der
Artikel dazu thematisiert die Bedenken, welche gegen das Buch geäussert
werden und auch, was diese Auszeichnung für eine negative Wirkung auf
die Wahrnehmung der Bibliothek als \enquote{safe space} und offener Ort
haben kann. Im Gegensatz zu den Verbotsdiskursen, die von
rechtsgerichteten Bewegungen bemüht werden, betonen aber hier alle
Beteiligten, dass es nicht darum gehe, das Buch grundsätzlich aus dem
Bibliotheksbestand zu entfernen, sondern es nicht gesondert
hervorzuheben. (ks)

\hypertarget{abschlussarbeiten}{%
\section{7. Abschlussarbeiten}\label{abschlussarbeiten}}

{[}Diesmal keine Beiträge{]}

\hypertarget{weitere-medien}{%
\section{8. Weitere Medien}\label{weitere-medien}}

BBK Berliner Bibliothekswissenschaftliches Kolloquium vom 15.11.2022,
Online. Prof.~Rebecca D. Frank, Ph.D.~(School of Information Sciences,
University of Tennessee, Knoxvill): \emph{Feminist Perspectives in Data
\& Information Science.}

In ihrem Vortrag zu feministischen Perspektiven in Daten- und
Informationswissenschaft legte Professorin Frank den Fokus auf
Datenrepositorien und Archivtheorien beziehungsweise Archive selbst. Sie
stellte Fragen, die an das Buch \enquote{Data Feminism} von Catherine
D'Ignazio und Lauren F. Klein (Die Buchrezension befindet sich in der
DLDL \#8.\footnote{Siehe Abschnitt 3.1. in \enquote{Das liest die LIBREAS,
  Nummer \#8 (Frühling / Sommer 2021)}. LIBREAS. Library Ideas, 39
  (2021). \url{https://libreas.eu/ausgabe39/dldl/}}) angelehnt sind,
wie: Wer erhebt Daten für wen? Welche Daten sind verfügbar und welche
nicht? Sie rief den Zuhörer*innen in Erinnerung, dass Archive,
Repositorien und andere Institutionen, welche Daten speichern, weder
neutral noch objektiv sind. Es gibt Regelwerke, Policies und nicht
zuletzt Personen, die darüber entscheiden, was es wert ist, archiviert
und publiziert zu werden. Wissenschaftler*innen sollten beim Thema Bias
also nicht nur die Entstehung der Daten im Blick haben, sondern auch die
Institutionen, in denen sie entstehen, gespeichert und publiziert
werden. Bei diesem Prozess sollte man sich die Frage stellen: Wer
entscheidet darüber, welche Daten es \enquote{wert} sind, gespeichert/
publiziert zu werden? Und für wen werden die Daten kuratiert, archiviert
und publiziert? Für wen nicht? Rebecca D. Frank fordert die
Zuhörer*innen dazu auf, über diese Themen nachzudenken, sich darüber
auszutauschen und das Feld der Bibliotheks- und Informationswissenschaft
unter diesem Gesichtspunkt zu betrachten. Sie regt dazu an, Gegebenes zu
hinterfragen, sich zu überlegen, wann, wie, wo Archive beziehungsweise
Repositorien entstanden sind. Welche Verzerrungen, Lücken und Vorurteile
könnten schon von Anfang an in die Systeme eingearbeitet worden sein?

Der Vortrag war sehr aufschlussreich und hat eine wichtige, aber
wahrscheinlich oft vergessene Perspektive auf Repositorien, Archive und
Forschungsdaten geworfen. Es ist zu hoffen, dass dieses BBK dazu
beigetragen hat, das Thema nachhaltig mehr in das Bewusstsein der
Teilnehmenden zu rücken. (sj)

\begin{center}\rule{0.5\linewidth}{0.5pt}\end{center}

Bernier, Charles L. (1968). \emph{Abstracts and Abstracting}. In: Kent,
Allen; Lancour, Harold (eds.): Encyclopedia of Library and Information
Science. Volume 1 A to Associac. New York: Marcel Dekker Inc, 1968.
S.16--38

Der Autor schließt seine Überblicksdarstellung zum Abstracting mit einem
Blick in die Zukunft von Abstracts. Dabei betont er an zwei Stellen die
Idee offener und kontinuierlich aktualisierter
Überblickszusammenstellung (\enquote{automated encyclopedia}), die die
jeweils aktuellen Daten entweder aus den Dokumenten oder bereits aus dem
Messkontext im Labor integrieren. Er schließt dabei ausdrücklich auch
Rohdaten mit ein und prophezeit, dass ein solches System nicht nur
Abstracts, sondern wissenschaftliche Aufsätze an sich überflüssig machen
würde. Die Forschenden wären nicht mehr Autor*innen, sondern menschliche
Datenverarbeiter (\enquote{human data processors}). (bk)

\begin{center}\rule{0.5\linewidth}{0.5pt}\end{center}

Alter, Alexandra; Harris, Elizabeth H. (2023): \emph{Efforts Double to
Ban Books in Schools and Libraries}. In: New York Times, March 27, 2023,
Section C, Page 6 / nytimes.com, 23.03.2023,
\url{https://www.nytimes.com/2023/03/23/books/book-ban-2022.html?searchResultPosition=2}

Die Auswertung einer aktuellen Erhebung der American Library Association
(ALA) berichtet von 1.269 Fällen, in denen im Jahr 2022 in den USA
versucht wurde, Bücher und andere Materialien aus Bibliotheken und
Schulen entfernen zu lassen. Das entspricht einem Höchststand für die
zurückliegenden 20 Jahre, wobei die Dunkelziffer als deutlich höher
eingeschätzt wird. Hinter den Vorgängen stehen konservative
Interessengruppen und Politiker*innen. Sie richten sich vorwiegend gegen
Medien zu Themen aus den Bereichen LGBTQ-Rechte, Geschlechtsidentität
und Rassismus. Benannt werden die Titel \enquote{The Bluest Eye} von
Toni Morrison, \enquote{The Handmaid's Tale} von Margaret Atwood sowie
\enquote{This Book is Gay} von Juno Dawson und \enquote{Gender Queer}
von Maia Kobabe. (bk)

\begin{center}\rule{0.5\linewidth}{0.5pt}\end{center}

Smith, Dana G. (2023): \emph{How to Focus Like It's 1990}. In: New York
Times / nytimes.com, 09.01.2023,
\url{https://www.nytimes.com/2023/01/09/well/mind/concentration-focus-distraction.html}

Verblüffenderweise gelten die 1990er-Jahre mittlerweile anscheinend als
das Shangri-La tiefer Konzentration. Die New York Times zieht sie
jedenfalls als Referenzrahmen heran und das auch belegt. Studien
stellten seit diesem Jahrzehnt eine schrumpfende Aufmerksamkeitsspanne
fest. Fokus ist also das, was es unter Einfluss von Push- und
Ping-Signalen wiederzuerlangen gilt. Drei Hinweise, dem
entgegenzuwirken, serviert der kurze Artikel: Erstens gilt es zu
verstehen, was die Ablenkung und Aufmerksamkeitssprünge triggert. Der
Drang, aufs Smartphone zu schauen, wird oft zum Muster, das man mit
kleinen Alltagslösungen zu durchbrechen versuchen kann. Das kann man
beispielsweise, zweitens, über \enquote{tech breaks} versuchen. Das
Smartphone bewusst 15 Minuten an der Seite liegenzulassen, kann zur
Entwöhnung und Stärkung der Selbstkontrolle beitragen. Der Zeitrahmen
lässt sich beliebig ausdehnen. Der dritte Punkt ist vielleicht der
interessanteste. Die Kultur- und Leseforscherin Maryanne Wolf rät zum
Lesen auf dem Papier, idealerweise als Deep Reading. Deep Reading
deshalb, da es offenbar zu einer Verschiebung kam, die viele Menschen
dazu bringt, auch auf Papier so sprunghaft scrollend zu lesen, wie sie
es vom Display gewohnt sind. Für den Einstieg werden täglich zwanzig
Minuten konzentrierte Lektüre eines Buches empfohlen. Maryanne Wolf hat
es geschafft, mit dieser Methode binnen zwei Wochen wieder die Kontrolle
über ihre Aufmerksamkeit zu erlangen und zurück zur Freude am Lesen zu
finden. (bk)

\begin{center}\rule{0.5\linewidth}{0.5pt}\end{center}

Ministerium für Hoch- und Fachschulwesen: \emph{Anordnung über die
Koordinierung der bibliothekswissenschaftlichen Forschung in der
Deutschen Demokratischen Republik.} In: Gesetzblatt der DDR. Berlin,
29.04.1977, S.142 f.

Im April 1977 wurden in einer Anordnung des Ministeriums für das Hoch-
und Fachschulwesen der DDR ein Rahmenplan zur Bibliothekswissenschaft in
der DDR durch den amtierenden Minister Hans-Joachim Böhme verkündet, der
das Institut für Bibliothekswissenschaft und wissenschaftliche
Information der Humboldt-Universität zu Berlin als
Koordinierungseinrichtung des Landes festschrieb (§ 3). Zugleich
definiert die Anordnung in den Grundsätzen (§ 2), was
bibliothekswissenschaftliche Forschung in diesem Kontext sei:

``(1) Die bibliothekswissenschaftliche Forschung dient dem Ziel,

\begin{itemize}
\item
  wissenschaftliche Grundlagen für die Leitung, Planung und Organisation
  der Bibliotheksarbeit auszuarbeiten,
\item
  wissenschaftlichen Vorlauf für die Erfordernisse der
  bibliothekarischen Praxis und für die Aus- und Weiterbildung der
  bibliothekarischen Fachkräfte zu schaffen,
\item
  die fortgeschrittensten Arbeitserfahrungen des Bibliotheks- und
  Informationswesens zu verallgemeinern und allen Bibliotheken
  zugänglich zu machen,
\item
  wissenschaftliche Erkenntnisse anderer Staaten, besonders der
  Sowjetunion und anderer sozialistischer Staaten, zu erschließen und zu
  verbreiten,
\item
  humanistische und sozialistische Traditionen der Bibliotheksgeschichte
  zu erschließen.''
\end{itemize}

Verallgemeinert wurden als Forschungsagenden der Bibliothekswissenschaft
also in etwa Bibliotheksverwaltung beziehungsweise
Bibliotheksmanagement, Kompetenzaufbau und -vermittlung,
Fachinformationsvermittlung, Erschließung und Vermittlung
internationaler Erkenntnisse sowie Bibliotheksgeschichte bestimmt. (bk)

\begin{center}\rule{0.5\linewidth}{0.5pt}\end{center}

Schulte, Joanna (2022): \emph{ZURÜCK / RETOUR / RETURN}. Berlin:
Revolver Publishing, 2022.

Für ihre Arbeit \enquote{An Oliver}, bei der die Künstlerin Joanna
Schulte Ersttagsbriefe der DDR mittels eines Return-to-Sender von
thematisch mehr oder weniger zum Motiv der jeweiligen Briefmarkenausgabe
passenden Orten verschickt und sammelt, sendete sie im Jahr 2019 den
Ersttagsbrief mit den am 18. August 1981 herausgegebenen Marken
\enquote{Kostbarkeiten in Bibliotheken der DDR} aus Marrakesch
(vergleiche S. 161\,f.). Sie frankierte den Brief mit einer am 15. März
2017 ausgegebenen selbstklebenden Briefmarke einer insgesamt zehn Werke
zum Motiv Mineralien umfassenden Ausgabe, in diesem Fall eine Briefmarke
mit dem Motiv Erythrin und einem Postwert von 9 \emph{drhm}. Die DDR
selbst hatte 1969 eine Marke mit diesem Motiv im Satz \enquote{Minerale
aus den Sammlungen der Bergakademie Freiberg} ausgegeben. Im
vorliegenden Fall liegt aber möglicherweise eine Verknüpfung mit dem auf
dem 20-Pfennig-Wert der Bibliotheksausgabe gezeigten Papyrus Ebers vor,
ein altägyptischer Papyrus, der zahlreiche medizinische Themen
behandelt, unter anderem zur Staublunge von Steinmetzen. Der Brief ging
an den Kunstverein in Wolfenbüttel. (bk)

%autor

\end{document}